\documentclass{article}
\usepackage[english,russian]{babel}
\usepackage{amsmath}
\usepackage{amsfonts}
\usepackage{amssymb}
\usepackage{geometry}
\usepackage{setspace}
\geometry{a4paper, portrait, margin=1in}

\title{Экзаменационная программа по курсу «Введение в математический анализ», осенний семестр 2020–2021 учебного года}
\date{}

\begin{document}
    \maketitle
    \pagenumbering{gobble}

    \newpage
    \pagenumbering{arabic}
    
    \onehalfspacing

    \section{}
    \subsection*{Действительные  числа}
        \subsubsection*{Дедекиндовы сечения}
        Сечение множества рациональных чисел $\mathbb{Q}$ $(A_*, A^*)$ -- разбиение $\mathbb{Q}$ на два таких непустых множества $A_*$ и $A^*$, таких, что:
        \begin{itemize}
            \item $A_* \cup A^* = \mathbb{Q}$
            \item $A_* \cap A^* = \varnothing$
            \item $\forall x \in A_*, \forall y \in A^* \longmapsto y > x$
        \end{itemize}
        \subsubsection*{Иррациональные числа}
        В сечении вида {\large \textcircled{\small A}} $A_*$ не имеет наибольшего элемента, а $A^*$ имеет наименьший.
        В сечении вида {\large \textcircled{\small B}} $A_*$ имеет наибольший элемент, а $A^*$ не имеет наименьшего.
        В сечении вида {\large \textcircled{\small C}} $A_*$ не имеет наибольшего элемента, а $A^*$ не имеет наименьшего.
        \\
        Иррациональным числом называется сечение вида \textcircled{C}.
        
        \subsubsection*{Действительные числа}
        Действительным числом называется любое сечение множества $\mathbb{Q}$ вида {\large \textcircled{\small A}} или {\large \textcircled{\small C}}.
        
        \subsubsection*{Упорядоченность, плотность и непрерывность действительных чисел}
        Пусть $\alpha, \beta \in \mathbb{R}, \alpha = (A_*, A^*), \beta = (B_*, B^*)$
        \\
        $\alpha = \beta$ если $A_* = B_*$
        \\     
        \\
        Предложение:
        \\
        Если $\alpha, \beta \in \mathbb{R}, \alpha \neq \beta$, то имеет место одно из включений: $A_* \subset B_*$ либо $A_* \supset B_*$
        \\
        \textbf{Доказательство:}
        пусть $A_* \not\subset B_*$ и $A_* \not\supset B_*$ 
        \\
        Тогда $\exists a \in \mathbb{Q}: a \in A_* \wedge a \notin B_* \implies a \in B^*$
        и $\exists b \in \mathbb{Q}: b \notin A_* \wedge b \in B_* \implies b \in A^*$
        \\
        так как $a \neq b$
        \\
        $a \in A_*, b \in A^* \implies b > a$
        \\
        $b \in B_*, a \in B^* \implies a > b$
        \\
        противоречие
        \\
        \\
        Пусть $\alpha, \beta \in \mathbb{R}, \alpha = (A_*, A^*), \beta = (B_*, B^*)$
        \\
        $\alpha < \beta$ если $A_* \neq B_* \wedge A_* \subset B_*$
        \\
        \\
        Упорядоченность $\mathbb{R}$:
        \\
        $\forall \alpha, \beta \in \mathbb{R}$ имеет место либо $\alpha = \beta$, либо $\alpha < \beta$,  либо $\alpha > \beta$.
        \\
        $ \alpha = \beta, \beta = \gamma \implies \alpha = \gamma$
        \\
        $ \alpha < \beta, \beta < \gamma \implies \alpha < \gamma$
        \\
        \\
        Плотность $\mathbb{Q}$ в $\mathbb{R}$:
        \\
        Пусть $\alpha, \beta \in \mathbb{R}, \alpha < \beta$, тогда $\exists c \in \mathbb{Q}: \alpha < c < \beta$
        \\
        \textbf{Доказательство:}
        $\alpha < \beta \implies A_* \subset B_* \implies \exists c \in \mathbb{Q}: c \in B_* \wedge c \notin A_*$.
        \\
        Так как в нижнем классе нет наибольшего элемента, $\alpha \le c < \beta$.
        \\
        Если $\alpha \in \mathbb{I}$, то $c \neq \alpha \implies \alpha < c < \beta$.
        Eсли $\alpha \in \mathbb{Q}$, то можно взять $c \in B_*: c > \alpha$. 
        \\
        \\
        Сечение множества действительных чисел $\mathbb{R}$ $(\mathcal{A_*}, \mathcal{A^*})$ --
        разбиение $\mathbb{Q}$ на два таких непустых множества $\mathcal{A_*}$ и $\mathcal{A^*}$, таких, что:
        \begin{itemize}
            \item $\mathcal{A_*} \cup \mathcal{A^*} = \mathbb{R}$
            \item $\mathcal{A_*} \cap \mathcal{A^*} = \varnothing$
            \item $\forall x \in \mathcal{A_*}, \forall y \in \mathcal{A^*} \longmapsto y > x$
        \end{itemize}
        \mbox{}
        \\
        Теорема Дедекинда:
        \\
        Среди сечений множества $\mathbb{R}$ сечений вида {\large \textcircled{\small C}} нет $\implies$ непрерывность $\mathbb{R}$.

        \subsubsection*{Десятичные дроби}
        Пусть числу $\alpha \in \mathbb{R}$ соответствует сечение $(\mathcal{A_*};\mathcal{A^*})$. За $a_0$ обозначим наибольшее целое число из $\mathcal{A_*}$.
        Отрезок $[a_0;a_0 + 1]$ поделим на $10$ отрезков одинаковой длины и выберем среди них тот, который содержит $\alpha$:
        $\alpha \in [a_0 + \frac{a_1}{10};a_0 + \frac{a_1 + 1}{10}]$.
        На шаге $n$ $\alpha \in [a_0 + \frac{a_1}{10} + ... + \frac{a_n}{10};a_0 + \frac{a_1}{10} + ... + \frac{a_n + 1}{10}]$.
        Бесконечную десятичную дробь $a_0,a_1a_2...a_n...$ можно считать представлением действительного числа $\alpha$.
        Заметим, что если $\alpha$ можно представить как $\frac{p}{10^n}, p \in \mathbb{Z}, n \in \mathbb{N}$, (то есть $\alpha$ -- сократимая десятичная дробь)
        то $\alpha$ соответствуют две десятичные дроби: $a_0,a_1a_2...a_n000...$ и $a_0,a_1a_2...(a_n - 1)999...$ .
        
        
    \subsection*{Теорема  о  существовании  и  единственности точной верхней (нижней) грани числового множества, ограниченного сверху (снизу)}
        Множество $X$ ограничено сверху, если $\exists C \in \mathbb{R}: \forall x \in X \longmapsto x \le C$
        \\
        \\
        Число $M$ называется верхней гранью множества $X$, если $\forall x \in X \longmapsto x \le M$
        \\
        \\
        Число $\alpha = \sup X$ называется точной верхней гранью множества $X$, если $\forall x \in X \longmapsto x \le \alpha \wedge \forall \alpha' \exists x \in X: x > \alpha'$.
        \\
        \\
        \textbf{Лемма:} если множество $X \subset \mathbb{R}$ имеет наибольший элемент $M = \max X$, то $M = \sup X$.
        \\
        \textbf{Доказательство:}
        так как $M$ -- наибольший элемент $X$, то все остальные элементы $X$ меньше $M$ и не являются верхней гранью $X$, так как
        $M \in X$ и $M > x$. Следовательно, $M$ -- точная верхняя грань $X$.
        \\
        \\
        \textbf{Теорема:} для ограниченного сверху множества $X \subset \mathbb{R}$ существует единственная точная верхняя грань.
        \\
        \textbf{Доказательство:}
        если в $X$ есть наибольший элемент, то $\alpha$ равно ему.
        Есть в $X$ наибольшего элемента нет, то построим сечение $(\mathcal{A_*}, \mathcal{A^*})$, такое, что в $\mathcal{A^*}$ содержатся все верхние грани $X$,
        а в $\mathcal{A_*}$ -- все остальные числа, при этом множество $\mathcal{A^*}$ не пусто, так как $X$ ограничено сверху,
        а $X \subset \mathcal{A_*}$, так как если элемент из $X \in \mathcal{A^*}$, то он является максимальным.
        По теореме Дедекинда, если либо больший элемент в $\mathcal{A_*}$, либо меньший в $\mathcal{A^*}$.
        Если в $\mathcal{A_*}$ есть наибольший элемент, то он является верхней гранью $X$ -- противоречие. Следовательно есть наименьший элемент в $\mathcal{A^*}$,
        который по определению является точной верхней гранью $X$.
        \\
        Теперь докажем единственность точной верхней грани.
        Пусть $\alpha, \alpha'$ -- точные верхние грани множества $X$, $\alpha' < \alpha$. Так как $\alpha$ -- точная верхняя грань,
        $\forall \beta < \alpha \exists x \in X: x > \beta \implies \exists x' \in X: x' > {\alpha'} \implies \alpha' \neq \sup X$.
        
        
    \subsection*{Счетность множества рациональных чисел}
        Докажем счетность множества полжительных рациональных чисел.
        \\ 
        Пусть $H \ge 2 \in \mathbb{N}$. Рассмотрим все взаимно простые пары чисел $p, q \in {N}$, такие что $p + q = H$, и соответствующие им рациональные числа.
        Понятно, что таких пар конечное число, и таким образом представляется любое рациональное число.
        \\
        Теперь расставим соответствующие каждому $H$ рациональные числа по порядку и пронумеруем их:
        \\
        $\frac{1}{1}, \frac{1}{2}, \frac{2}{1}, \frac{1}{3}, \frac{3}{1}, \frac{1}{4}, \frac{2}{3}, \frac{3}{2}, \frac{4}{1}, ...$
        \\
        Так как множество положительных рациональных чисел счетно, то и аналогичным образом счетно множество отрицательных рациональных чисел,
        а их объединение в объединении вместе с конечным множетсвом, состоящим из $0$, так же счетно и является $\mathbb{R}$.
        
        
    \subsection*{Несчетность множества действительных чисел}
        Если подмножество множества несчетно, то и само оно несчетно.
        \\
        Рассмотрим числа на интервале $(0;1)$, представленные в виде десятичных дробей:
        \\
        $\alpha_1 = 0,a_1^1a_2^1...a_n^1...$
        \\
        $\alpha_2 = 0,a_1^2a_2^2...a_n^2...$
        \\
        ...
        \\
        $\alpha_k = 0,a_1^ka_2^k...a_n^k...$
        \\
        ...
        \\
        Допустим, что подмножество $(0;1)$ счетно.
        \\
        Построим число $\gamma$ такое, что $\gamma = 0,c_1c_2...c_n...$; $c_i \neq a_i^i, c_i \neq 9$. $\gamma$ не равно ни одному из $a_i$,
        что противоречит тому, что $(0;1)$ счетно. Следовательно, само множество действительных чисел также счетно.
        
    \newpage

    \section{}
    \subsection*{Бесконечно малые и бесконечно большие последовательности и их свойства}
        Последовательность $\{x_n\}$ называется бесконечно большой, если:
        \[ \forall M > 0 \exists N = N(M): \forall n \ge N \longmapsto |x_n| > M \]
        Последовательность $\{x_n\}$ называется бесконечно малой, если:
        \[ \forall \epsilon > 0 \exists N = N(\epsilon): \forall n \ge N \longmapsto |x_n| < \epsilon \]
        \textbf{Теорема:} сумма двух бесконечно маллых последовательностей также является бесконечно малой.
        \\
        \textbf{Доказательство:}
        пусть последовательности $\{x_n\}$ и $\{y_n\}$ -- бесконечно малые:
        \[ \forall \epsilon > 0 \exists N_x = N_x(\epsilon): \forall n \ge N_x \longmapsto |x_n| < \frac{\epsilon}{2} \]
        \[ \forall \epsilon > 0 \exists N_y = N_y(\epsilon): \forall n \ge N_y \longmapsto |y_n| < \frac{\epsilon}{2} \]
        \[ \forall \epsilon > 0 \exists N = \max (N_x(\epsilon), N_y(\epsilon)): \forall n \ge N \longmapsto |x_n| < \frac{\epsilon}{2} \wedge |y_n| < \frac{\epsilon}{2} \]
        \[ |x_n + y_n| \le |x_n| + |y_n| < \frac{\epsilon}{2} + \frac{\epsilon}{2} = \epsilon \]
        следовательно, $\{x_n\} + \{y_n\}$ -- также бесконечно малая последовательность.
        \\
        \\
        \textbf{Теорема:} бесконечно малая последовательность ограничена.
        \\
        \textbf{Доказательство:}
        пусть последовательность $\{x_n\}$ -- бесконечно малая:
        \[ \forall \epsilon > 0 \exists N = N(\epsilon): \forall n \ge N \longmapsto |x_n| < \epsilon \]
        Возьмем произвольное $\epsilon_0$. Тогда $\forall n \ge N(\epsilon_0) \longmapsto |x_n| < \epsilon_0$.
        Выберем среди первых $N(\epsilon_0)$ членов последовательности максимальный по модулю и обозначим его модуль за $\epsilon_1$.
        Тогда $\forall x: 1 \le x \le N(\epsilon_0) \longmapsto |x_n| \le \epsilon_1$.
        Следовательно, $\forall n \in \mathbb{N} \longmapsto |x_n| \le \epsilon = \max(\epsilon_0, \epsilon_1) \implies \{x_n\}$ ограничена.
        \\
        \\
        \textbf{Теорема:} произведение бесконечно малой и ограниченной последовательностей -- бесконечно малая последовательность.
        \\
        \textbf{Доказательство:}
        пусть $\{x_n\}$ -- ограниченная последовательность, $\{y_n\}$ -- бесконечно малая:
        \[ \exists C > 0: \forall n \longmapsto |x_n| \le C \]
        \[ \forall \epsilon > 0 \exists N = N(\epsilon): \forall n \ge N \longmapsto |y_n| < \frac{\epsilon}{C} \]
        \[ \forall \epsilon > 0 \exists N = N(\epsilon): \forall n \ge N \longmapsto |x_ny_n| \le C|y_n| < \epsilon \]
        \\
        \\
        \textbf{Теорема:} произведение двух бесконечно малых последовательностей -- бесконечно малая последовательность.
        \\
        \textbf{Доказательство:}
        Любая бесконечно малая последовательность ограничена, следовательно произведение двух бесконечно малых последовательностей -- бесконечно
        малая, как и произведение бесконечно малой и ограниченной.
        \\
        \\
        \textbf{Теорема:} если все члены бесконечно малой последовательности с какого-то номера равны $\gamma$, то $\gamma = 0$.
        \\
        \textbf{Доказательство:}
        пусть $\gamma \neq 0$. Последовательность $\{x_n\}$ бесконечно мала, возьмем $\epsilon_0 = |\gamma|$ и проверим для него условие:
        \[ \exists N = N(\gamma): \forall n \ge N \longmapsto |\gamma| < |\gamma| \]
        получается противоречие.
        \\
        \\
        \textbf{Теорема:} если последовательность $\{x_n\}$ -- бесконечно большая, то с какого-то номера определена бесконечно малая последовательность $\{y_n\} = \{\frac{1}{x_n}\}$.
        \\
        \textbf{Доказательство:}
        так как $\{x_n\}$ -- бесконечно большая:
        \[ \forall M > 0 \exists N = N(M): \forall n \ge N \longmapsto |x_n| > \frac{1}{M} \implies \frac{1}{|x_n|} < {M}\]
        Следовательно $\{y_n\}$ -- бесконечно малая.
        \\
        \\
        \textbf{Теорема:} если последовательность $\{x_n\}$ -- ограниченная, а $\{y_n\}$ -- бесконечно большая,
        то с какого-то номера определена бесконечно малая последовательность $\{z_n\} = \{\frac{x_n}{y_n}\}$.
        \\
        \textbf{Доказательство:}
        так как $\{y_n\}$ -- бесконечно большая, $\{y'_n\}$, где $y'_n = \frac{1}{y_n}$, -- бесконечно малая. Тогда $\{z_n\}$ -- бесконечно малая как
        произведение бесконечно малой и ограниченной.
        \\
        \\
        \textbf{Теорема:} если последовательность $\{|x_n|\}$ ограничена снизу $c > 0$, а $\{y_n\}$ -- бесконечно малая и $y_n \neq 0 \forall n$,
        то $\{z_n\} = \{\frac{x_n}{y_n}\}$ -- бесконечно большая.
        \\
        \textbf{Доказательство:}
        так как $\{y_n\}$ -- бесконечно малая:
        \[ \forall \epsilon > 0 \exists N = N(\epsilon): \forall n \ge N \longmapsto |y_n| < \frac{1}{\epsilon} \implies \frac{1}{|y_n|} > {\epsilon}\]
        Следовательно $\{y_n\}$ -- бесконечно большая. $ |x_ny_n| > C|y_n| > \epsilon $, следовательно $\{z_n\}$ -- бесконечно большая.
    
    
    \subsection*{Предел числовой последовательности}
        Последовательность $\{x_n\}$ сходится к $a$, если последовательность $\{y_n\}$, где $y_n = x_n - a$, бесконечно мала.
        \\
        Число $a = \lim_{x\to\infty} x_n$ называется пределом последовательности $\{x_n\}$, если:
        \\
        \[ \forall \epsilon > 0 \exists N = N(\epsilon): \forall n \ge N \longmapsto |x_n - a| < \epsilon \]
    
    
    \subsection*{Единственность предела}
        \textbf{Доказательство:}
        пусть $a_1$ -- предел последовательности $\{x_n\}$ и $a_2$ -- предел $\{x_n\}$. Тогда:
        \[ \forall \epsilon > 0 \exists N_1 = N_1(\epsilon): \forall n \ge N_1 \longmapsto |x_n - a_1| < \epsilon \]
        \[ \forall \epsilon > 0 \exists N_2 = N_2(\epsilon): \forall n \ge N_2 \longmapsto |x_n - a_2| < \epsilon \]
        \[ \forall \epsilon > 0 \exists N = \max (N_1(\epsilon), N_2(\epsilon)): \forall n \ge N \longmapsto
        x_n \in (a_1 - \epsilon; a_1 + \epsilon) \cap (a_2 - \epsilon; a_2 + \epsilon) \]
        При $\epsilon \le \frac{a_1 + a_2}{2}$ $(a_1 - \epsilon; a_1 + \epsilon) \cap (a_2 - \epsilon; a_2 + \epsilon) = \varnothing$ -- противоречие.
    
    \subsection*{Арифметические операции со сходящимися последовательностями}
        \textbf{Теорема:} если последовательность $\{x_n\}$ сходится к $a$, а $\{y_n\}$ -- к $b$, то их сумма $\{z_n\} = \{x_n\} + \{y_n\}$ сходится к $a + b$.
        \\
        \textbf{Доказательство:}
        $x_n = a_n + a$, $y_n = b_n + b$, где $\{a_n\}$ и $\{b_n\}$ -- бесконечно малые последовательности.
        Последовательность $\{c_n\}$, где $c_n = x_n - a + y_n - b = a_n + b_n$, -- бесконечно мала, следовательно
        $a + b$ -- предел $\{z_n\}$
        \\
        \\
        \textbf{Теорема:} если последовательность $\{x_n\}$ сходится к $a$, а $\{y_n\}$ -- к $b$, то их произведение $\{z_n\} = \{x_n\} \cdot \{y_n\}$ сходится к $ab$.
        \\
        \textbf{Доказательство:}
        $x_n = a_n + a$, $y_n = b_n + b$, где $\{a_n\}$ и $\{b_n\}$ -- бесконечно малые последовательности.
        Последовательность $\{c_n\}$, где $c_n = x_n y_n - ab = (a_n + a) (b_n + b) - ab = a_n b_n + b a_n + a b_n $, -- бесконечно мала, следовательно
        $ab$ -- предел $\{z_n\}$
        \\
        \\
        \textbf{Теорема:} если последовательность $\{x_n\}$ сходится к $a$, а $\{y_n\}$ -- к $b \neq 0$, то с какого-то номера
        определена последовательность их частного $\{z_n\} = \frac{\{x_n\}}{\{y_n\}}$, которая сходится к $\frac{a}{b}$.    
        \\
        \textbf{Доказательство:}
        $x_n = a_n + a$, $y_n = b_n + b$, где $\{a_n\}$ и $\{b_n\}$ -- бесконечно малые последовательности.
        \\
        Докажем сначала существование $\{z_n\}$. Пусть $\epsilon_0 = \frac{|b|}{2}$.
        \[ \exists N = N(\epsilon_0): \forall n \ge N \longmapsto |y_n - b| < \frac{|b|}{2} \]
        \[ \frac{|b|}{2} < y_n < \frac{3|b|}{2} \]
        \[ \frac{1}{y_n} < \frac{2}{|b|} \]
        Видно, что $\{y_n\}$ с $N(\epsilon_0)$ не равна $0$, и что последовательность $\{\frac{1}{y_n}\}$ ограничена.
        \\
        Последовательность $\{c_n\}$, где $c_n = \frac{x_n}{y_n} - \frac{a}{b} = \frac{a_n + a}{b_n + b} - \frac{a}{b} =
        \frac{b a_n + ab - ab - a b_n}{b(b_n + b)} = \frac{1}{b_n + b}(a_n - \frac{a}{b} b_n)$, -- бесконечно мала как произведение ограниченной
        и бесконечно малой, следовательно $\frac{a}{b}$ -- предел $\{z_n\}$
    
    \subsection*{Свойства  пределов,  связанные  с неравенствами}
        \textbf{Теорема:} Если для последовательности $\{x_n\}$
        $\exists n_0 \in \mathbb{N}, b \in \mathbb{R}: \forall n \ge n_0 \longmapsto x_n \ge b \wedge \exists \lim_{n\to\infty} x_n = a$, то $a \ge b$.
        \\
        \textbf{Доказательство:}
        Пуcть $a < b$:
        \[ \epsilon_0 = b - a \]
        \[ \exists n_1 = n_1(\epsilon_0) \ge n_0: \forall n \ge n_1 \longmapsto |x_n - a| < b - a \]
        \[ x_n - a < b - a \]
        \[ x_n < b \]
        что противоречит тому, что $x_n \ge b $
        \\
        \\
        \textbf{Теорема:} Если для последовательностей $\{x_n\}$ и $\{y_n\}$
        $\exists n_0: \forall n \ge n_0 \longmapsto x_n \ge y_n \wedge \exists \lim_{n\to\infty} x_n = a \wedge \exists \lim_{n\to\infty} y_n = b$, то $a \ge b$.
        \\
        \textbf{Доказательство:}
        Пуcть $a < b$:
        \[ \epsilon_0 = \frac{b - a}{2} \]
        \[ \exists n_1 = n_1(\epsilon_0) \ge n_0: \forall n \ge n_1 \longmapsto |x_n - a| < \frac{b - a}{2} \]
        \[ \exists n_2 = n_2(\epsilon_0) \ge n_0: \forall n \ge n_2 \longmapsto |y_n - b| < \frac{b - a}{2} \]
        \[ x_n < a + \frac{b - a}{2} = \frac{b + a}{2} \]
        \[ b - \frac{b - a}{2} = \frac{b + a}{2} < y_n \]
        \[ x_n < y_n \]
        что противоречит тому, что $x_n \ge y_n $
        \\
        \\
        \textbf{Теорема:} Если для последовательностей $\{x_n\}$, $\{y_n\}$ и $\{z_n\}$
        то $\exists \lim_{n\to\infty} z_n = a$.
        $\exists n_0: \forall n \ge n_0 \longmapsto  \wedge \exists \lim_{n\to\infty} x_n = \lim_{n\to\infty} y_n = a$,
        \\
        \textbf{Доказательство:}
        \[ x_n - a \ge y_n - a \ge z_n - a \]
        \[ |y_n - a| \le \max(|x_n - a|, |z_n - a|) \]
        \[ \forall \epsilon > 0 \exists N_1 = N_1(\epsilon_0): \forall n \ge N_1 \longmapsto |x_n - a| < \epsilon \]
        \[ \forall \epsilon > 0 \exists N_2 = N_2(\epsilon_0): \forall n \ge N_2 \longmapsto |z_n - a| < \epsilon \]
        \[ \forall \epsilon > 0 \exists N = \max(N_1(\epsilon_0), N_2(\epsilon_0)): \forall n \ge N_2 \longmapsto |y_n - a| < \epsilon \]
    
    \subsection*{Теорема  Вейерштрасса  о  пределе  монотонной  ограниченной  последовательности}
        \textbf{Теорема:} Если последовательность $\{x_n\}$ ограничена и монотонна, то она сходится к своей точной верхней грани (если она неубывает или возрастает),
        или к своей точной нижней грани (если она невозрастает или убывает).
        \\
        \textbf{Доказательство:}
        пусть $\{x_n\}$ возрастает. 
        \[ a = \sup \{x_n\} \implies \forall \epsilon \exists N = N(\epsilon): x_{N} > a - \epsilon \]
        \[ 0 \le a - x_{N} < \epsilon \]
        так как $\{x_n\}$ возрастает
        \[ \forall n > N \longmapsto x_n > x_N \implies 0 \le a - x_{n} < a - x_{N} < \epsilon \]
        следовательно $\{x_n\}$ сходится к $a = \sup \{x_n\}$.
        
        
    \subsection*{Теорема Кантора о вложенных отрезках}
        Системой стягивающихся отрезков называют последовательность $\{[a_n;b_n]\}$, если:
        \begin{itemize}
            \item $\forall n \longmapsto [a_{n+1};b_{n+1}] \subseteq [a_{n};b_{n}]$
            \item $\lim_{n\to\infty} (b_n - a_n) = 0$
        \end{itemize}
        \textbf{Теорема:} система стягивающихся отрезков имеет единственную точку, принадлежащую им всем.
        \\
        \textbf{Доказательство:}
        $\{a_n\}$, $\{b_n\}$ ограничены и монотонны, следовательно $\exists c = \sup \{a_n\} = \inf \{b_n\}$.
        \\
        Пусть $\exists c' > c: c' \in [a_n; b_n] \forall n$. Тогда $\forall n b_n - a_n \ge c' - c > 0 \implies lim_{n\to\infty} (b_n - a_n) > 0$,
        что противоречит тому, что $\lim_{n\to\infty} (b_n - a_n) = 0$.
        
        
    \subsection*{Число $e$}
        \textbf{Теорема:} число $e = \lim_{n\to\infty} \left(1 + \frac{1}{n} \right)^n$
        \\
        \textbf{Доказательство:}
        \[ x_n = 1 + n\frac{1}{n} + \frac{n(n-1)}{2!}\frac{1}{n^2} + ... +
        \frac{n(n-1)...(n - (n - 1)}{n!}\frac{1}{n^n} \]
        
        \[ x_n = 1 + 1 + \frac{1}{2!} \left(1 - \frac{1}{n}\right) + ... +
        \frac{1}{n!} \left(1 - \frac{1}{n}\right) \left(1 - \frac{2}{n}\right) ... \left(1 - \frac{n - 1}{n}\right) \]
        
        \[ x_{n + 1} = 1 + 1 + \frac{1}{2!} \left(1 - \frac{1}{n + 1}\right) + ... +
        \frac{1}{(n + 1)!} \left(1 - \frac{1}{n + 1}\right) \left(1 - \frac{2}{n + 1}\right) ... \left(1 - \frac{n}{n + 1}\right) \]
        
        \[ x_{n + 1} > x_{n} \]
        
        \[ \frac{1}{n!} < \frac{1}{2^{n-1}} \]
        
        \[ x_n \le 2 + \frac{1}{2!} + ... + \frac{1}{n!} \le 2 + \frac{1}{2} + ... + \frac{1}{2^{n-1}} =
        2 + 1 - \frac{1}{2^{n-1}} < 3 \]
        
        Последовательность монотонно возрастает и ограничена, следовательно $\exists \lim_{n\to\infty} \{x_n\} = e$.
        
        
    \newpage
        
    \section{}
    \subsection*{Подпоследовательности, частичные пределы}
        \subsubsection*{Подпоследовательности}
            Пусть $\{x_n\}$ -- произвольная последовательность, $\{k_n\}$ -- возрастающая последовательность из натуральных чисел.
            Тогда $\{x_{k_n}\}$ называется подпоследовательностью $\{x_n\}$.
            
        \subsubsection*{Частичные пределы}
            Если $\{x_{k_n}\}$ сходится к $a$, то $a$ -- частичный предел $\{x_n\}$.
        
        
    \subsection*{Теорема Больцано-Вейерштрасса}
        \textbf{Теорема:} если последовательность $\{x_n\}$ ограничена, то у нее есть сходящаяся последовательность $\{x_{k_n}\}$.
        \\
        \textbf{Доказательство:}
        Так как $\{x_n\}$ ограничена, $\exists \alpha, \beta: \alpha \le x_n \le \beta \forall n$.
        На отрезке $[\alpha; \beta]$ лежит бесконечное колличество членов последовательности.
        Выберем такой отрезок $[\alpha_1; \beta_1]$, на котором лежит бесконечное колличество членов последовательности
        и для которого $\beta_1 - \alpha_1 = \frac{\beta - \alpha}{2}$.
        На шаге $n$ имеем $\beta_n - \alpha_n = \frac{\beta_{n-1} - \alpha_{n-1}}{2}$.
        \[ \lim_{n\to\infty} (b_n - a_n) = \lim_{n\to\infty} (b - a)\frac{1}{2^n} = 0 \]
        \[ \lim_{n\to\infty} a_n = \lim_{n\to\infty} b_n = \gamma \]
        Выберем $x_{k_1}$ такое, что $x_{k_1} \in [a_1;b_1]$.
        Выберем $x_{k_2}$ такое, что $x_{k_2} \in [a_2;b_2]$ и $k_2 > k_1$. Такое $k_2$ существует, потому что если не учитывать все элементы
        последовательности, чей номер меньше или равен $k_1$, на $[a_2;b_2]$ все равно лежит бесконечное колличество членов $\{x_n\}$.
        На шаге $n$ выбираем $x_{k_n}$ такое, что $x_{k_n} \in [a_n;b_n]$ и $k_n > k_{n-1}$.
        \[ a_n \le x_{k_n} \le b_n \]
        \[ \lim_{n\to\infty} a_n = \lim_{n\to\infty} b_n = \gamma \implies \lim_{n\to\infty} x_{k_n} = \gamma \]
        
    
    \subsection*{Критерий Коши существования конечного предела последовательности}
        \subsubsection*{Фундаментальная последовательность}
        Если для последовательности $\{x_n\}$ выполняется следующее условие:
        \[ \forall \epsilon > 0 \exists N = N(\epsilon): \forall n,m \ge N \longmapsto |x_n - x_m| < \epsilon \]
        то она называется фундаментальной.
        
        \subsubsection*{Критерий Коши}
        \textbf{Теорема:} чтобы у последовательности $\{x_n\}$ существовал конечный предел, необходимо и достаточно,
        чтобы она была фундаментальной.
        \\
        \textbf{Доказательство:}
        \\
        Необходимость: пусть $\lim_{n\to\infty} x_n = a$. Тогда:
        \[ \forall \epsilon > 0 \exists N = N(\epsilon): \forall n, m \ge N \longmapsto |x_n - a| < \frac{\epsilon}{2} \wedge |x_m - a| < \frac{\epsilon}{2} \]
        \[ \frac{-\epsilon}{2} < x_n - a < \frac{\epsilon}{2} \]
        \[ \frac{-\epsilon}{2} < a - x_m < \frac{\epsilon}{2} \]
        \[ -\epsilon < x_n - x_m < \epsilon \]
        \[ |x_n - x_m| < \epsilon \]
        следовательно, $\{x_n\}$ -- фундаментальная.
        \\
        Достаточность: если $\{x_n\}$ фундаментальная, то она ограничена, следовательно у нее есть подпоследовательность $\{x_{k_n}\}$, сходящаяся к $a$:
        \[ \forall \epsilon > 0 \exists N_1 = N_1(\epsilon): \forall n \ge N_1 \longmapsto |x_{k_n} - a| < \frac{\epsilon}{2} \]
        Так как $\{x_n\}$ фундаментальная:
        \[ \forall \epsilon > 0 \exists N_2 = N_2(\epsilon): \forall n,m \ge N_2 \longmapsto |x_n - x_m| < \frac{\epsilon}{2} \]
        \[ \forall \epsilon > 0 \exists N = \max(N_1(\epsilon), N_2(\epsilon)): \forall n \ge N \longmapsto |x_n - a| = |x_n - x_{k_N} + x_{k_N} - a|
        \le |x_n - x_{k_N}| + |x_{k_N} - a| < \frac{\epsilon}{2} + \frac{\epsilon}{2} = \epsilon \]
        следовательно, $\{x_n\}$ сходится к $a$.
        
        
    \newpage

\end{document}
