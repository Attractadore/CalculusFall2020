\documentclass{article}
\usepackage[english,russian]{babel}
\usepackage{amsmath}
\usepackage{amsfonts}
\usepackage{amssymb}
\usepackage{geometry}
\geometry{a4paper, portrait, margin=1in}
\usepackage{setspace}
\usepackage{color}
\usepackage{hyperref}
\hypersetup{
    colorlinks=true,
    linktoc=all,
    linkcolor=blue,
}

\title{Экзаменационная программа по курсу «Введение в математический анализ», осенний семестр 2020–2021 учебного года}
\date{}

\newcommand{\theorem}{\textbf{Теорема:} }
\newcommand{\lemma}{\textbf{Лемма:} }
\newcommand{\proof}{\textbf{Доказательство:} }
\newcommand{\seq}[1]{$\{#1\}$}
\newcommand{\intr}[2]{$(#1;#2)$}
\newcommand{\otr}[2]{$[#1;#2]$}
\newcommand{\Q}{$\mathbb{Q}$}

\begin{document}
    \maketitle
    \pagenumbering{gobble}

    \newpage
    \pagenumbering{arabic}

    \onehalfspacing

    \setcounter{tocdepth}{3}
    \tableofcontents

    \newpage

    \section{}
    \subsection*{Действительные  числа}
        \subsubsection*{Дедекиндовы сечения}
        Сечение множества рациональных чисел $\mathbb{Q}$ $(A_*, A^*)$ -- разбиение $\mathbb{Q}$ на два таких непустых множества $A_*$ и $A^*$, таких, что:
        \begin{itemize}
            \item $A_* \cup A^* = \mathbb{Q}$
            \item $A_* \cap A^* = \varnothing$
            \item $\forall x \in A_*, \forall y \in A^* \longmapsto y > x$
        \end{itemize}
        \subsubsection*{Иррациональные числа}
        В сечении вида {\large \textcircled{\small A}} $A_*$ не имеет наибольшего элемента, а $A^*$ имеет наименьший.
        В сечении вида {\large \textcircled{\small B}} $A_*$ имеет наибольший элемент, а $A^*$ не имеет наименьшего.
        В сечении вида {\large \textcircled{\small C}} $A_*$ не имеет наибольшего элемента, а $A^*$ не имеет наименьшего.
        \\
        Иррациональным числом называется сечение вида \textcircled{C}.
        
        \subsubsection*{Действительные числа}
        Действительным числом называется любое сечение множества $\mathbb{Q}$ вида {\large \textcircled{\small A}} или {\large \textcircled{\small C}}.
        
        \subsubsection*{Упорядоченность, плотность и непрерывность действительных чисел}
        Пусть $\alpha, \beta \in \mathbb{R}, \alpha = (A_*, A^*), \beta = (B_*, B^*)$
        \\
        $\alpha = \beta$ если $A_* = B_*$
        \\     
        \\
        Предложение:
        \\
        Если $\alpha, \beta \in \mathbb{R}, \alpha \neq \beta$, то имеет место одно из включений: $A_* \subset B_*$ либо $A_* \supset B_*$
        \\
        \textbf{Доказательство:}
        пусть $A_* \not\subset B_*$ и $A_* \not\supset B_*$ 
        \\
        Тогда $\exists a \in \mathbb{Q}: a \in A_* \wedge a \notin B_* \implies a \in B^*$
        и $\exists b \in \mathbb{Q}: b \notin A_* \wedge b \in B_* \implies b \in A^*$
        \\
        так как $a \neq b$
        \\
        $a \in A_*, b \in A^* \implies b > a$
        \\
        $b \in B_*, a \in B^* \implies a > b$
        \\
        противоречие
        \\
        \\
        Пусть $\alpha, \beta \in \mathbb{R}, \alpha = (A_*, A^*), \beta = (B_*, B^*)$
        \\
        $\alpha < \beta$ если $A_* \neq B_* \wedge A_* \subset B_*$
        \\
        \\
        Упорядоченность $\mathbb{R}$:
        \\
        $\forall \alpha, \beta \in \mathbb{R}$ имеет место либо $\alpha = \beta$, либо $\alpha < \beta$,  либо $\alpha > \beta$.
        \\
        $ \alpha = \beta, \beta = \gamma \implies \alpha = \gamma$
        \\
        $ \alpha < \beta, \beta < \gamma \implies \alpha < \gamma$
        \\
        \\
        Плотность $\mathbb{Q}$ в $\mathbb{R}$:
        \\
        Пусть $\alpha, \beta \in \mathbb{R}, \alpha < \beta$, тогда $\exists c \in \mathbb{Q}: \alpha < c < \beta$
        \\
        \textbf{Доказательство:}
        $\alpha < \beta \implies A_* \subset B_* \implies \exists c \in \mathbb{Q}: c \in B_* \wedge c \notin A_*$.
        \\
        Так как в нижнем классе нет наибольшего элемента, $\alpha \le c < \beta$.
        \\
        Если $\alpha \in \mathbb{I}$, то $c \neq \alpha \implies \alpha < c < \beta$.
        Eсли $\alpha \in \mathbb{Q}$, то можно взять $c \in B_*: c > \alpha$. 
        \\
        \\
        Сечение множества действительных чисел $\mathbb{R}$ $(\mathcal{A_*}, \mathcal{A^*})$ --
        разбиение $\mathbb{Q}$ на два таких непустых множества $\mathcal{A_*}$ и $\mathcal{A^*}$, таких, что:
        \begin{itemize}
            \item $\mathcal{A_*} \cup \mathcal{A^*} = \mathbb{R}$
            \item $\mathcal{A_*} \cap \mathcal{A^*} = \varnothing$
            \item $\forall x \in \mathcal{A_*}, \forall y \in \mathcal{A^*} \longmapsto y > x$
        \end{itemize}
        \mbox{}
        \\
        Теорема Дедекинда:
        \\
        Среди сечений множества $\mathbb{R}$ сечений вида {\large \textcircled{\small C}} нет $\implies$ непрерывность $\mathbb{R}$.

        \subsubsection*{Десятичные дроби}
        Пусть числу $\alpha \in \mathbb{R}$ соответствует сечение $(\mathcal{A_*};\mathcal{A^*})$. За $a_0$ обозначим наибольшее целое число из $\mathcal{A_*}$.
        Отрезок $[a_0;a_0 + 1]$ поделим на $10$ отрезков одинаковой длины и выберем среди них тот, который содержит $\alpha$:
        $\alpha \in [a_0 + \frac{a_1}{10};a_0 + \frac{a_1 + 1}{10}]$.
        На шаге $n$ $\alpha \in [a_0 + \frac{a_1}{10} + ... + \frac{a_n}{10};a_0 + \frac{a_1}{10} + ... + \frac{a_n + 1}{10}]$.
        Бесконечную десятичную дробь $a_0,a_1a_2...a_n...$ можно считать представлением действительного числа $\alpha$.
        Заметим, что если $\alpha$ можно представить как $\frac{p}{10^n}, p \in \mathbb{Z}, n \in \mathbb{N}$, (то есть $\alpha$ -- сократимая десятичная дробь)
        то $\alpha$ соответствуют две десятичные дроби: $a_0,a_1a_2...a_n000...$ и $a_0,a_1a_2...(a_n - 1)999...$ .
        
        
    \subsection*{Теорема  о  существовании  и  единственности точной верхней (нижней) грани числового множества, ограниченного сверху (снизу)}
        Множество $X$ ограничено сверху, если $\exists C \in \mathbb{R}: \forall x \in X \longmapsto x \le C$
        \\
        \\
        Число $M$ называется верхней гранью множества $X$, если $\forall x \in X \longmapsto x \le M$
        \\
        \\
        Число $\alpha = \sup X$ называется точной верхней гранью множества $X$, если $\forall x \in X \longmapsto x \le \alpha \wedge \forall \alpha' \exists x \in X: x > \alpha'$.
        \\
        \\
        \textbf{Лемма:} если множество $X \subset \mathbb{R}$ имеет наибольший элемент $M = \max X$, то $M = \sup X$.
        \\
        \textbf{Доказательство:}
        так как $M$ -- наибольший элемент $X$, то все остальные элементы $X$ меньше $M$ и не являются верхней гранью $X$, так как
        $M \in X$ и $M > x$. Следовательно, $M$ -- точная верхняя грань $X$.
        \\
        \\
        \textbf{Теорема:} для ограниченного сверху множества $X \subset \mathbb{R}$ существует единственная точная верхняя грань.
        \\
        \textbf{Доказательство:}
        если в $X$ есть наибольший элемент, то $\alpha$ равно ему.
        Есть в $X$ наибольшего элемента нет, то построим сечение $(\mathcal{A_*}, \mathcal{A^*})$, такое, что в $\mathcal{A^*}$ содержатся все верхние грани $X$,
        а в $\mathcal{A_*}$ -- все остальные числа, при этом множество $\mathcal{A^*}$ не пусто, так как $X$ ограничено сверху,
        а $X \subset \mathcal{A_*}$, так как если элемент из $X \in \mathcal{A^*}$, то он является максимальным.
        По теореме Дедекинда, если либо больший элемент в $\mathcal{A_*}$, либо меньший в $\mathcal{A^*}$.
        Если в $\mathcal{A_*}$ есть наибольший элемент, то он является верхней гранью $X$ -- противоречие. Следовательно есть наименьший элемент в $\mathcal{A^*}$,
        который по определению является точной верхней гранью $X$.
        \\
        Теперь докажем единственность точной верхней грани.
        Пусть $\alpha, \alpha'$ -- точные верхние грани множества $X$, $\alpha' < \alpha$. Так как $\alpha$ -- точная верхняя грань,
        $\forall \beta < \alpha \exists x \in X: x > \beta \implies \exists x' \in X: x' > {\alpha'} \implies \alpha' \neq \sup X$.
        
        
    \subsection*{Счетность множества рациональных чисел}
        Докажем счетность множества полжительных рациональных чисел.
        \\ 
        Пусть $H \ge 2 \in \mathbb{N}$. Рассмотрим все взаимно простые пары чисел $p, q \in {N}$, такие что $p + q = H$, и соответствующие им рациональные числа.
        Понятно, что таких пар конечное число, и таким образом представляется любое рациональное число.
        \\
        Теперь расставим соответствующие каждому $H$ рациональные числа по порядку и пронумеруем их:
        \\
        $\frac{1}{1}, \frac{1}{2}, \frac{2}{1}, \frac{1}{3}, \frac{3}{1}, \frac{1}{4}, \frac{2}{3}, \frac{3}{2}, \frac{4}{1}, ...$
        \\
        Так как множество положительных рациональных чисел счетно, то и аналогичным образом счетно множество отрицательных рациональных чисел,
        а их объединение в объединении вместе с конечным множетсвом, состоящим из $0$, так же счетно и является $\mathbb{R}$.
        
        
    \subsection*{Несчетность множества действительных чисел}
        Если подмножество множества несчетно, то и само оно несчетно.
        \\
        Рассмотрим числа на интервале $(0;1)$, представленные в виде десятичных дробей:
        \\
        $\alpha_1 = 0,a_1^1a_2^1...a_n^1...$
        \\
        $\alpha_2 = 0,a_1^2a_2^2...a_n^2...$
        \\
        ...
        \\
        $\alpha_k = 0,a_1^ka_2^k...a_n^k...$
        \\
        ...
        \\
        Допустим, что подмножество $(0;1)$ счетно.
        \\
        Построим число $\gamma$ такое, что $\gamma = 0,c_1c_2...c_n...$; $c_i \neq a_i^i, c_i \neq 9$. $\gamma$ не равно ни одному из $a_i$,
        что противоречит тому, что $(0;1)$ счетно. Следовательно, само множество действительных чисел также счетно.
        
    \newpage

    \section{}
    \subsection*{Бесконечно малые и бесконечно большие последовательности и их свойства}
        Последовательность $\{x_n\}$ называется бесконечно большой, если:
        \[ \forall M > 0 \exists N = N(M): \forall n \ge N \longmapsto |x_n| > M \]
        Последовательность $\{x_n\}$ называется бесконечно малой, если:
        \[ \forall \epsilon > 0 \exists N = N(\epsilon): \forall n \ge N \longmapsto |x_n| < \epsilon \]
        \textbf{Теорема:} сумма двух бесконечно маллых последовательностей также является бесконечно малой.
        \\
        \textbf{Доказательство:}
        пусть последовательности $\{x_n\}$ и $\{y_n\}$ -- бесконечно малые:
        \[ \forall \epsilon > 0 \exists N_x = N_x(\epsilon): \forall n \ge N_x \longmapsto |x_n| < \frac{\epsilon}{2} \]
        \[ \forall \epsilon > 0 \exists N_y = N_y(\epsilon): \forall n \ge N_y \longmapsto |y_n| < \frac{\epsilon}{2} \]
        \[ \forall \epsilon > 0 \exists N = \max (N_x(\epsilon), N_y(\epsilon)): \forall n \ge N \longmapsto |x_n| < \frac{\epsilon}{2} \wedge |y_n| < \frac{\epsilon}{2} \]
        \[ |x_n + y_n| \le |x_n| + |y_n| < \frac{\epsilon}{2} + \frac{\epsilon}{2} = \epsilon \]
        следовательно, $\{x_n\} + \{y_n\}$ -- также бесконечно малая последовательность.
        \\
        \\
        \textbf{Теорема:} бесконечно малая последовательность ограничена.
        \\
        \textbf{Доказательство:}
        пусть последовательность $\{x_n\}$ -- бесконечно малая:
        \[ \forall \epsilon > 0 \exists N = N(\epsilon): \forall n \ge N \longmapsto |x_n| < \epsilon \]
        Возьмем произвольное $\epsilon_0$. Тогда $\forall n \ge N(\epsilon_0) \longmapsto |x_n| < \epsilon_0$.
        Выберем среди первых $N(\epsilon_0)$ членов последовательности максимальный по модулю и обозначим его модуль за $\epsilon_1$.
        Тогда $\forall x: 1 \le x \le N(\epsilon_0) \longmapsto |x_n| \le \epsilon_1$.
        Следовательно, $\forall n \in \mathbb{N} \longmapsto |x_n| \le \epsilon = \max(\epsilon_0, \epsilon_1) \implies \{x_n\}$ ограничена.
        \\
        \\
        \textbf{Теорема:} произведение бесконечно малой и ограниченной последовательностей -- бесконечно малая последовательность.
        \\
        \textbf{Доказательство:}
        пусть $\{x_n\}$ -- ограниченная последовательность, $\{y_n\}$ -- бесконечно малая:
        \[ \exists C > 0: \forall n \longmapsto |x_n| \le C \]
        \[ \forall \epsilon > 0 \exists N = N(\epsilon): \forall n \ge N \longmapsto |y_n| < \frac{\epsilon}{C} \]
        \[ \forall \epsilon > 0 \exists N = N(\epsilon): \forall n \ge N \longmapsto |x_ny_n| \le C|y_n| < \epsilon \]
        \\
        \\
        \textbf{Теорема:} произведение двух бесконечно малых последовательностей -- бесконечно малая последовательность.
        \\
        \textbf{Доказательство:}
        Любая бесконечно малая последовательность ограничена, следовательно произведение двух бесконечно малых последовательностей -- бесконечно
        малая, как и произведение бесконечно малой и ограниченной.
        \\
        \\
        \textbf{Теорема:} если все члены бесконечно малой последовательности с какого-то номера равны $\gamma$, то $\gamma = 0$.
        \\
        \textbf{Доказательство:}
        пусть $\gamma \neq 0$. Последовательность $\{x_n\}$ бесконечно мала, возьмем $\epsilon_0 = |\gamma|$ и проверим для него условие:
        \[ \exists N = N(\gamma): \forall n \ge N \longmapsto |\gamma| < |\gamma| \]
        получается противоречие.
        \\
        \\
        \textbf{Теорема:} если последовательность $\{x_n\}$ -- бесконечно большая, то с какого-то номера определена бесконечно малая последовательность $\{y_n\} = \{\frac{1}{x_n}\}$.
        \\
        \textbf{Доказательство:}
        так как $\{x_n\}$ -- бесконечно большая:
        \[ \forall M > 0 \exists N = N(M): \forall n \ge N \longmapsto |x_n| > \frac{1}{M} \implies \frac{1}{|x_n|} < {M}\]
        Следовательно $\{y_n\}$ -- бесконечно малая.
        \\
        \\
        \textbf{Теорема:} если последовательность $\{x_n\}$ -- ограниченная, а $\{y_n\}$ -- бесконечно большая,
        то с какого-то номера определена бесконечно малая последовательность $\{z_n\} = \{\frac{x_n}{y_n}\}$.
        \\
        \textbf{Доказательство:}
        так как $\{y_n\}$ -- бесконечно большая, $\{y'_n\}$, где $y'_n = \frac{1}{y_n}$, -- бесконечно малая. Тогда $\{z_n\}$ -- бесконечно малая как
        произведение бесконечно малой и ограниченной.
        \\
        \\
        \textbf{Теорема:} если последовательность $\{|x_n|\}$ ограничена снизу $c > 0$, а $\{y_n\}$ -- бесконечно малая и $y_n \neq 0 \forall n$,
        то $\{z_n\} = \{\frac{x_n}{y_n}\}$ -- бесконечно большая.
        \\
        \textbf{Доказательство:}
        так как $\{y_n\}$ -- бесконечно малая:
        \[ \forall \epsilon > 0 \exists N = N(\epsilon): \forall n \ge N \longmapsto |y_n| < \frac{1}{\epsilon} \implies \frac{1}{|y_n|} > {\epsilon}\]
        Следовательно $\{y_n\}$ -- бесконечно большая. $ |x_ny_n| > C|y_n| > \epsilon $, следовательно $\{z_n\}$ -- бесконечно большая.
    
    
    \subsection*{Предел числовой последовательности}
        Последовательность $\{x_n\}$ сходится к $a$, если последовательность $\{y_n\}$, где $y_n = x_n - a$, бесконечно мала.
        \\
        Число $a = \lim_{x\to\infty} x_n$ называется пределом последовательности $\{x_n\}$, если:
        \\
        \[ \forall \epsilon > 0 \exists N = N(\epsilon): \forall n \ge N \longmapsto |x_n - a| < \epsilon \]
    
    
    \subsection*{Единственность предела}
        \textbf{Доказательство:}
        пусть $a_1$ -- предел последовательности $\{x_n\}$ и $a_2$ -- предел $\{x_n\}$. Тогда:
        \[ \forall \epsilon > 0 \exists N_1 = N_1(\epsilon): \forall n \ge N_1 \longmapsto |x_n - a_1| < \epsilon \]
        \[ \forall \epsilon > 0 \exists N_2 = N_2(\epsilon): \forall n \ge N_2 \longmapsto |x_n - a_2| < \epsilon \]
        \[ \forall \epsilon > 0 \exists N = \max (N_1(\epsilon), N_2(\epsilon)): \forall n \ge N \longmapsto
        x_n \in (a_1 - \epsilon; a_1 + \epsilon) \cap (a_2 - \epsilon; a_2 + \epsilon) \]
        При $\epsilon \le \frac{a_1 + a_2}{2}$ $(a_1 - \epsilon; a_1 + \epsilon) \cap (a_2 - \epsilon; a_2 + \epsilon) = \varnothing$ -- противоречие.
    
    \subsection*{Арифметические операции со сходящимися последовательностями}
        \textbf{Теорема:} если последовательность $\{x_n\}$ сходится к $a$, а $\{y_n\}$ -- к $b$, то их сумма $\{z_n\} = \{x_n\} + \{y_n\}$ сходится к $a + b$.
        \\
        \textbf{Доказательство:}
        $x_n = a_n + a$, $y_n = b_n + b$, где $\{a_n\}$ и $\{b_n\}$ -- бесконечно малые последовательности.
        Последовательность $\{c_n\}$, где $c_n = x_n - a + y_n - b = a_n + b_n$, -- бесконечно мала, следовательно
        $a + b$ -- предел $\{z_n\}$
        \\
        \\
        \textbf{Теорема:} если последовательность $\{x_n\}$ сходится к $a$, а $\{y_n\}$ -- к $b$, то их произведение $\{z_n\} = \{x_n\} \cdot \{y_n\}$ сходится к $ab$.
        \\
        \textbf{Доказательство:}
        $x_n = a_n + a$, $y_n = b_n + b$, где $\{a_n\}$ и $\{b_n\}$ -- бесконечно малые последовательности.
        Последовательность $\{c_n\}$, где $c_n = x_n y_n - ab = (a_n + a) (b_n + b) - ab = a_n b_n + b a_n + a b_n $, -- бесконечно мала, следовательно
        $ab$ -- предел $\{z_n\}$
        \\
        \\
        \textbf{Теорема:} если последовательность $\{x_n\}$ сходится к $a$, а $\{y_n\}$ -- к $b \neq 0$, то с какого-то номера
        определена последовательность их частного $\{z_n\} = \frac{\{x_n\}}{\{y_n\}}$, которая сходится к $\frac{a}{b}$.    
        \\
        \textbf{Доказательство:}
        $x_n = a_n + a$, $y_n = b_n + b$, где $\{a_n\}$ и $\{b_n\}$ -- бесконечно малые последовательности.
        \\
        Докажем сначала существование $\{z_n\}$. Пусть $\epsilon_0 = \frac{|b|}{2}$.
        \[ \exists N = N(\epsilon_0): \forall n \ge N \longmapsto |y_n - b| < \frac{|b|}{2} \]
        \[ \frac{|b|}{2} < y_n < \frac{3|b|}{2} \]
        \[ \frac{1}{y_n} < \frac{2}{|b|} \]
        Видно, что $\{y_n\}$ с $N(\epsilon_0)$ не равна $0$, и что последовательность $\{\frac{1}{y_n}\}$ ограничена.
        \\
        Последовательность $\{c_n\}$, где $c_n = \frac{x_n}{y_n} - \frac{a}{b} = \frac{a_n + a}{b_n + b} - \frac{a}{b} =
        \frac{b a_n + ab - ab - a b_n}{b(b_n + b)} = \frac{1}{b_n + b}(a_n - \frac{a}{b} b_n)$, -- бесконечно мала как произведение ограниченной
        и бесконечно малой, следовательно $\frac{a}{b}$ -- предел $\{z_n\}$
    
    \subsection*{Свойства  пределов,  связанные  с неравенствами}
        \textbf{Теорема:} Если для последовательности $\{x_n\}$
        $\exists n_0 \in \mathbb{N}, b \in \mathbb{R}: \forall n \ge n_0 \longmapsto x_n \ge b \wedge \exists \lim_{n\to\infty} x_n = a$, то $a \ge b$.
        \\
        \textbf{Доказательство:}
        Пуcть $a < b$:
        \[ \epsilon_0 = b - a \]
        \[ \exists n_1 = n_1(\epsilon_0) \ge n_0: \forall n \ge n_1 \longmapsto |x_n - a| < b - a \]
        \[ x_n - a < b - a \]
        \[ x_n < b \]
        что противоречит тому, что $x_n \ge b $
        \\
        \\
        \textbf{Теорема:} Если для последовательностей $\{x_n\}$ и $\{y_n\}$
        $\exists n_0: \forall n \ge n_0 \longmapsto x_n \ge y_n \wedge \exists \lim_{n\to\infty} x_n = a \wedge \exists \lim_{n\to\infty} y_n = b$, то $a \ge b$.
        \\
        \textbf{Доказательство:}
        Пуcть $a < b$:
        \[ \epsilon_0 = \frac{b - a}{2} \]
        \[ \exists n_1 = n_1(\epsilon_0) \ge n_0: \forall n \ge n_1 \longmapsto |x_n - a| < \frac{b - a}{2} \]
        \[ \exists n_2 = n_2(\epsilon_0) \ge n_0: \forall n \ge n_2 \longmapsto |y_n - b| < \frac{b - a}{2} \]
        \[ x_n < a + \frac{b - a}{2} = \frac{b + a}{2} \]
        \[ b - \frac{b - a}{2} = \frac{b + a}{2} < y_n \]
        \[ x_n < y_n \]
        что противоречит тому, что $x_n \ge y_n $
        \\
        \\
        \textbf{Теорема:} Если для последовательностей $\{x_n\}$, $\{y_n\}$ и $\{z_n\}$
        то $\exists \lim_{n\to\infty} z_n = a$.
        $\exists n_0: \forall n \ge n_0 \longmapsto  \wedge \exists \lim_{n\to\infty} x_n = \lim_{n\to\infty} y_n = a$,
        \\
        \textbf{Доказательство:}
        \[ x_n - a \ge y_n - a \ge z_n - a \]
        \[ |y_n - a| \le \max(|x_n - a|, |z_n - a|) \]
        \[ \forall \epsilon > 0 \exists N_1 = N_1(\epsilon_0): \forall n \ge N_1 \longmapsto |x_n - a| < \epsilon \]
        \[ \forall \epsilon > 0 \exists N_2 = N_2(\epsilon_0): \forall n \ge N_2 \longmapsto |z_n - a| < \epsilon \]
        \[ \forall \epsilon > 0 \exists N = \max(N_1(\epsilon_0), N_2(\epsilon_0)): \forall n \ge N_2 \longmapsto |y_n - a| < \epsilon \]
    
    \subsection*{Теорема  Вейерштрасса  о  пределе  монотонной  ограниченной  последовательности}
        \textbf{Теорема:} Если последовательность $\{x_n\}$ ограничена и монотонна, то она сходится к своей точной верхней грани (если она неубывает или возрастает),
        или к своей точной нижней грани (если она невозрастает или убывает).
        \\
        \textbf{Доказательство:}
        пусть $\{x_n\}$ возрастает. 
        \[ a = \sup \{x_n\} \implies \forall \epsilon \exists N = N(\epsilon): x_{N} > a - \epsilon \]
        \[ 0 \le a - x_{N} < \epsilon \]
        так как $\{x_n\}$ возрастает
        \[ \forall n > N \longmapsto x_n > x_N \implies 0 \le a - x_{n} < a - x_{N} < \epsilon \]
        следовательно $\{x_n\}$ сходится к $a = \sup \{x_n\}$.
        
        
    \subsection*{Теорема Кантора о вложенных отрезках}
        Системой стягивающихся отрезков называют последовательность $\{[a_n;b_n]\}$, если:
        \begin{itemize}
            \item $\forall n \longmapsto [a_{n+1};b_{n+1}] \subseteq [a_{n};b_{n}]$
            \item $\lim_{n\to\infty} (b_n - a_n) = 0$
        \end{itemize}
        \textbf{Теорема:} система стягивающихся отрезков имеет единственную точку, принадлежащую им всем.
        \\
        \textbf{Доказательство:}
        $\{a_n\}$, $\{b_n\}$ ограничены и монотонны, следовательно $\exists c = \sup \{a_n\} = \inf \{b_n\}$.
        \\
        Пусть $\exists c' > c: c' \in [a_n; b_n] \forall n$. Тогда $\forall n b_n - a_n \ge c' - c > 0 \implies lim_{n\to\infty} (b_n - a_n) > 0$,
        что противоречит тому, что $\lim_{n\to\infty} (b_n - a_n) = 0$.
        
        
    \subsection*{Число $e$}
        \textbf{Теорема:} число $e = \lim_{n\to\infty} \left(1 + \frac{1}{n} \right)^n$
        \\
        \textbf{Доказательство:}
        \[ x_n = 1 + n\frac{1}{n} + \frac{n(n-1)}{2!}\frac{1}{n^2} + ... +
        \frac{n(n-1)...(n - (n - 1)}{n!}\frac{1}{n^n} \]
        
        \[ x_n = 1 + 1 + \frac{1}{2!} \left(1 - \frac{1}{n}\right) + ... +
        \frac{1}{n!} \left(1 - \frac{1}{n}\right) \left(1 - \frac{2}{n}\right) ... \left(1 - \frac{n - 1}{n}\right) \]
        
        \[ x_{n + 1} = 1 + 1 + \frac{1}{2!} \left(1 - \frac{1}{n + 1}\right) + ... +
        \frac{1}{(n + 1)!} \left(1 - \frac{1}{n + 1}\right) \left(1 - \frac{2}{n + 1}\right) ... \left(1 - \frac{n}{n + 1}\right) \]
        
        \[ x_{n + 1} > x_{n} \]
        
        \[ \frac{1}{n!} < \frac{1}{2^{n-1}} \]
        
        \[ x_n \le 2 + \frac{1}{2!} + ... + \frac{1}{n!} \le 2 + \frac{1}{2} + ... + \frac{1}{2^{n-1}} =
        2 + 1 - \frac{1}{2^{n-1}} < 3 \]
        
        Последовательность монотонно возрастает и ограничена, следовательно $\exists \lim_{n\to\infty} \{x_n\} = e$.
        
        
    \newpage
        
    \section{}
    \subsection*{Подпоследовательности, частичные пределы}
        \subsubsection*{Подпоследовательности}
            Пусть $\{x_n\}$ -- произвольная последовательность, $\{k_n\}$ -- возрастающая последовательность из натуральных чисел.
            Тогда $\{x_{k_n}\}$ называется подпоследовательностью $\{x_n\}$.
            
        \subsubsection*{Частичные пределы}
            Если $\{x_{k_n}\}$ сходится к $a$, то $a$ -- частичный предел $\{x_n\}$.
        
        
    \subsection*{Теорема Больцано-Вейерштрасса}
        \textbf{Теорема:} если последовательность $\{x_n\}$ ограничена, то у нее есть сходящаяся последовательность $\{x_{k_n}\}$.
        \\
        \textbf{Доказательство:}
        Так как $\{x_n\}$ ограничена, $\exists \alpha, \beta: \alpha \le x_n \le \beta \forall n$.
        На отрезке $[\alpha; \beta]$ лежит бесконечное колличество членов последовательности.
        Выберем такой отрезок $[\alpha_1; \beta_1]$, на котором лежит бесконечное колличество членов последовательности
        и для которого $\beta_1 - \alpha_1 = \frac{\beta - \alpha}{2}$.
        На шаге $n$ имеем $\beta_n - \alpha_n = \frac{\beta_{n-1} - \alpha_{n-1}}{2}$.
        \[ \lim_{n\to\infty} (b_n - a_n) = \lim_{n\to\infty} (b - a)\frac{1}{2^n} = 0 \]
        \[ \lim_{n\to\infty} a_n = \lim_{n\to\infty} b_n = \gamma \]
        Выберем $x_{k_1}$ такое, что $x_{k_1} \in [a_1;b_1]$.
        Выберем $x_{k_2}$ такое, что $x_{k_2} \in [a_2;b_2]$ и $k_2 > k_1$. Такое $k_2$ существует, потому что если не учитывать все элементы
        последовательности, чей номер меньше или равен $k_1$, на $[a_2;b_2]$ все равно лежит бесконечное колличество членов $\{x_n\}$.
        На шаге $n$ выбираем $x_{k_n}$ такое, что $x_{k_n} \in [a_n;b_n]$ и $k_n > k_{n-1}$.
        \[ a_n \le x_{k_n} \le b_n \]
        \[ \lim_{n\to\infty} a_n = \lim_{n\to\infty} b_n = \gamma \implies \lim_{n\to\infty} x_{k_n} = \gamma \]
        
    
    \subsection*{Критерий Коши существования конечного предела последовательности}
        \subsubsection*{Фундаментальная последовательность}
        Если для последовательности $\{x_n\}$ выполняется следующее условие:
        \[ \forall \epsilon > 0 \exists N = N(\epsilon): \forall n,m \ge N \longmapsto |x_n - x_m| < \epsilon \]
        то она называется фундаментальной.
        
        \subsubsection*{Критерий Коши}
        \textbf{Теорема:} чтобы у последовательности $\{x_n\}$ существовал конечный предел, необходимо и достаточно,
        чтобы она была фундаментальной.
        \\
        \textbf{Доказательство:}
        \\
        Необходимость: пусть $\lim_{n\to\infty} x_n = a$. Тогда:
        \[ \forall \epsilon > 0 \exists N = N(\epsilon): \forall n, m \ge N \longmapsto |x_n - a| < \frac{\epsilon}{2} \wedge |x_m - a| < \frac{\epsilon}{2} \]
        \[ \frac{-\epsilon}{2} < x_n - a < \frac{\epsilon}{2} \]
        \[ \frac{-\epsilon}{2} < a - x_m < \frac{\epsilon}{2} \]
        \[ -\epsilon < x_n - x_m < \epsilon \]
        \[ |x_n - x_m| < \epsilon \]
        следовательно, $\{x_n\}$ -- фундаментальная.
        \\
        Достаточность: если $\{x_n\}$ фундаментальная, то она ограничена, следовательно у нее есть подпоследовательность $\{x_{k_n}\}$, сходящаяся к $a$:
        \[ \forall \epsilon > 0 \exists N_1 = N_1(\epsilon): \forall n \ge N_1 \longmapsto |x_{k_n} - a| < \frac{\epsilon}{2} \]
        Так как $\{x_n\}$ фундаментальная:
        \[ \forall \epsilon > 0 \exists N_2 = N_2(\epsilon): \forall n,m \ge N_2 \longmapsto |x_n - x_m| < \frac{\epsilon}{2} \]
        \[ \forall \epsilon > 0 \exists N = \max(N_1(\epsilon), N_2(\epsilon)): \forall n \ge N \longmapsto |x_n - a| = |x_n - x_{k_N} + x_{k_N} - a|
        \le |x_n - x_{k_N}| + |x_{k_N} - a| < \frac{\epsilon}{2} + \frac{\epsilon}{2} = \epsilon \]
        следовательно, $\{x_n\}$ сходится к $a$.
        
        
    \newpage
    
    \section{}
    \subsection*{Определения предела функции в терминах окрестностей и в терминах последовательностей, их эквивалентность}
        Далее подразумевается, что функция $f(x)$ определена в некой выколотой $\Delta$ окрестности точки $x_0$: $X = \{x: 0 < |x - x_0| < \Delta\}$.
        \subsubsection*{Предел функции в точке по Коши}
        Число $a = \lim_{x \to x_0} f(x)$ называется пределом функции $f(x)$ в точке $x_0$, если:
        \[ \forall \epsilon > 0 \exists \delta = \delta(\epsilon)> 0: \forall x \in X: 0 < |x - x_0| < \delta \longmapsto |f(x) - f(x_0)| < \epsilon \]
    
    \subsubsection*{Предел функции а точке по Гейне}
        Последовательностью Гейне назовем сходящуюся к $x_0$ последовательность $\{x_n\}$, где $x_n \neq x_0$.
        \\
        Число $a$ называется пределом функции в точке $x_0$, если для произвольной сходящейся к $x_0$ последовательности Гейне с какого-то $n_0$ определена $f(x_n)$,
        и $\lim_{n \to \infty} f(x_n) = a$
    
    \subsubsection*{Эквивалентность определений предела по Коши и по Гейне}
        \textbf{Теорема:} определения предела функции в точке по Коши и по Гейне эквивалентны.
        \\
        \textbf{Доказательство:}
        \\
        Пусть $a$ -- предел функции в точке $x_0$ по Коши:
        \[ \forall \epsilon > 0 \exists \delta = \delta(\epsilon): \forall x \in X: 0 < |x_n - x_0| < \delta \]
        Для любой последовательности Гейне, сходящейся к $x_0$, найдется $N = N(\delta) = N(\epsilon): 
        \forall n \ge N \longmapsto 0 < |x_n - x_0| < \delta \implies |f(x_n) - a| < \epsilon \implies \lim_{n \to \infty} f(x_n) = a$. $f(x_n)$ сходится к $a$ для произвольной последовательности
        Гейне, следовательно $a$ -- предел $f(x)$ в точке $x_0$ по Гейне.
        \\
        Пусть $a$ -- предел функции в точке $x_0$ по Гейне, а по Коши он не равен $a$:
        \[ \exists \epsilon_0 > 0: \forall \delta > 0 \exists x_\delta: 0 < |x_\delta - x_0| < \delta \longmapsto |f(x_\delta) - a| \ge \epsilon_0 \]
        \[\delta_n = \frac{1}{n} > 0 \implies \exists x_n: 0 < |x_n - x_0| < \delta_n \longmapsto |f(x_n) - a| \ge \epsilon_0 \]
        То есть существует такая сходящаяся к $x_0$ последовательность Гейне $\{x_n\}$, для которой $\{f(x_n)\}$ не сходится к $a$, что противоречит тому, что $a$ -- предел по Гейне.  
    
    \subsection*{Свойства пределов функции}
        Вытекают из свойств предела последовательностей и применения определения предела функции по Гейне.
    
    \subsection*{Критерий Коши существования конечного предела функции}
        Функция $f(x)$ удовлетворяет условию Коши в точке $x_0$, если:
        \[ \forall \epsilon > 0 \exists \delta = \delta(\epsilon): \forall x_1, x_2 \in X: 0 < |x_1 - x_0| < \delta \wedge 0 < |x_2 - x_0| < \delta \longmapsto
        |f(x_1) - f(x_2)| < \epsilon \]
        \textbf{Теорема:} для того, чтобы функция $f(x)$ в имела конечный предел в точке $x_0$, необходимо и достаточно, чтобы она удовлетворяла условию Коши в точке $x_0$.
        \\
        \textbf{Доказательство:}
        \\
        Необходимость: пусть $a = \lim_{x \to x_0} f(x)$:
        \[ \forall \epsilon > 0 \exists \delta = \delta(\epsilon) > 0: \forall x_1, x_2 \in X: 0 < |x_1 - x_0| < \delta \wedge 0 < |x_2 - x_0| < \delta \longmapsto
        |(f(x_1) - f(x_0)| < \frac{\epsilon}{2} \wedge |(f(x_0) - f(x_2)| < \frac{\epsilon}{2} \]
        \[ -\frac{\epsilon}{2} < f(x_1) - f(x_0) < \frac{\epsilon}{2} \]
        \[ -\frac{\epsilon}{2} < f(x_0) - f(x_2) < \frac{\epsilon}{2} \]
        \[ -\epsilon < f(x_1) - f(x_2) < \epsilon \]
        \[ |f(x_1) - f(x_2)| < \epsilon \]
        следовательно, $f(x)$ удовлетворяет условию Коши в точке $x_0$.
        \\
        Достаточность: пусть $f(x)$ удовлетворяет условию Коши в точке $x_0$:
        \[ \forall \epsilon > 0 \exists \delta = \delta(\epsilon): \forall x_1, x_2 \in X: 0 < |x_1 - x_0| < \delta \wedge 0 < |x_2 - x_0| < \delta \longmapsto
        |f(x_1) - f(x_2)| < \epsilon \]
        Возьмем произвольную сходящуюся к $x_0$ последовательность Гейне.
        \[ \forall \delta \exists N = N(\delta) = N(\epsilon): \forall n, m \ge N \longmapsto 0 < |x_n - x_0| < \delta \wedge 0 < |x_m - x_0| < \delta \longmapsto
        |f(x_n) - f(x_m)| < \epsilon \]
        Следовательно $\{f(x_n)\}$ фундаментальна и сходится к некому $a$.
        \\
        Покажем, что $\{f(x_n)\}$ сходится к одному $a$ вне зависимости от выбора последовательности Гейне $\{x_n\}$.
        \\
        Пусть $\{x'_n\}$ и $\{x''_n\}$ сходятся к $x_0$, $\{f(x'_n)\}$ сходится к $a'$, $\{f(x''_n)\}$ сходится к $a''$.
        Построим новую последовательность Гейне $\{\bar{x}_n\} = \{x'_1, x''_1, x'_2, x''_2, ..., x'_n, x''_n, ... \}$. $\{f(\bar{x}_n)\}$ -- фундаментальная, значит она сходится
        к $\bar{a}$. Раз $\{f(\bar{x}_n)\}$ сходится к $\bar{a}$, значит $\{f(x'_n)\}$ и $a'$, $\{f(x''_n)\}$ сходятся к $\bar{a}$, $\bar{a} = a' = a''$.
        Следовательно $\bar{a}$ -- предел $f(x)$ в точке $x_0$ по Гейне. 
    
    \subsection*{Предел сложной функции}
        Пусть функция $f(x)$ определена в выколотой $\Delta_1$ окрестности точки $x_0$: $X = \{x: 0 < |x - x_0| < \Delta_1\}$,
        а функция $g(x)$ определена в выколотой $\Delta_2$ окрестности точки $y_0$: $Y = \{y: 0 < |y - y_0| < \Delta_2\}$,
        $f(X) \subseteq Y$.
        \\
        \textbf{Теорема:} Если предел функции $f(x)$ в точке $x_0$ равен $y_0$, сама $f(x)$ на $X$ не принимает значения $y_0$, предел функции $g(y)$ в точке $y_0$ равен $a$,
        то предел функции $h(x) = g(f(x))$ в точке $x_0$ равен $a$.
        \\
        \textbf{Доказательство:}
        \\
        Для произвольной сходящейся к $x_0$ последовательности Гейне $\{x_n\}$ $\{f(x_n)\}$ является сходящейся к $y_0$ последовательностью Гейне.
        Следовательно, $\{g(f(x_n))\} = \{h(x_n)\}$ сходится к $a$, $a$ -- предел в точке $x_0$ функции $h(x)$ про Гейне.
    
    
    \subsection*{Существование односторонних пределов у монотонных функций}
        \textbf{Теорема:} если $f(x)$ определена и монотонна на $(a;b)$, то $ \forall x_0 \in (a;b) \exists f(x_0 - 0) \wedge \exists f(x_0 + 0)$.
        Если $f(x)$ неубывает или возрастает на $(a;b)$, то $f(x_0 - 0) \le f(x_0) \le f(x_0 + 0)$.
        Если $f(x)$ невозрастает или убывает на $(a;b)$, то $f(x_0 - 0) \ge f(x_0) \ge f(x_0 + 0)$.
        \\
        \textbf{Доказательство:} пусть $f(x)$ возрастает на $(a;b)$.
        Докажем существование $f(x - 0) \forall x$:
        \[ \forall x_0 \in (a;b) \forall x \in (a;x_0) \longmapsto f(x) < f(x_0) \implies \exists \sup_{(a;x_0)} f(x) = \alpha \le f(x_0)\]
        \[ \forall \epsilon \exists x_\epsilon \in (a;x_0): f(x_\epsilon) > \alpha - \epsilon \]
        \[ \delta = \delta(\epsilon) = x_0 - x_\epsilon > 0 \]
        \[ \forall x: 0 < x_0 - x < \delta \longmapsto \alpha - \epsilon < f(x) \le \alpha\]
        \[ \alpha - \epsilon < f(x) < \alpha + \epsilon \]
        \[ |f(x) - \alpha| < \epsilon \]
        \[ f(x_0 - 0) = \alpha \]
        существование $f(x + 0) \forall x$ доказывается аналогичным образом.


    \newpage

    \section{}

    \subsection*{Непрерывность функции в точке}
        Функция $f(x)$ называет непрерывной в точке $x_0$, если 
        \[ \lim_{x \to x_0} f(x) = f(x_0) \]


    \subsection*{Свойства функций, непрерывных в точке}
        \textbf{Теорема:} если функции $f(x)$ и $g(x)$ непрерывны в точке $x_0$, то функции $f(x) + g(x)$ и $f(x)g(x)$ тоже непрерывны в $x_0$.
        Если $g(x_0) \neq 0$, то функция $\frac{f(x)}{g(x)}$ непрерывна в точке $x_0$.
        \\
        \textbf{Доказательство:} 
        Следует из свойств арифметических операций с пределами.
        \\
        \\
        \textbf{Теорема:} если функция $f(x)$ определена и непрерывна в точке $x_0$, то существует такая $\delta$-окрестность точки $x_0$,
        в пределах которой $f(x)$ сохраняет свой знак.
        \\
        \\
        Функцию $\Delta y(x_0, \Delta x)$ в точке $x_0$ определим как $f(x_0 + \Delta x) - f(x_0)$.
        \\
        \textbf{Теорема:} если функция $f(x)$ непрерывна в точке $x_0$, то $\lim_{\Delta x \to 0} \Delta y(x_0, \Delta x) = 0$.  
        
         
    \subsection*{Односторонняя непрерывность}
        Функция $f(x)$ называет одностонне непрерывной в точке $x_0$, если 
        \[ \lim_{x \to x_0 \pm 0} f(x) = f(x_0) \]
        
        
    \subsection*{Непрерывность сложной функции}
        \textbf{Теорема:} если функция $f(x)$ непрерывна в точке $x_0$, а функция $g(y)$ непрерывна в точке $y_0 = f(x_0)$, то функция $h(x) = g(f(x_0))$ непрерывна
        в точке $x_0$.
        \\
        \textbf{Доказательство:} по теореме о переходе к пределу под знаком непрерывной функции $\lim_{x \to x_0} g(f(x)) = g(y_0) = g(f(x_0))$.
        
    \subsection*{Теорема о переходе к пределу под знаком непрерывной функции}       
        \textbf{Теорема:} если предел функции $f(x)$ точке $x_0$ равен $y_0$, а функция $g(y)$ непрерывна в точке $y_0 = f(x_0)$, то
        $\lim_{x \to x_0} g(f(x)) = g \left(\lim_{x \to x_0} f(x) \right)$.
        \\
        \textbf{Доказательство:}
        возьмем сходящуюся к $x_0$ последовательность Гейне $\{x_n\}$, тогда последовательность $\{y_n\} = \{f(x_n)\}$ сходится к $y_0$.
        Последовательность $\{g(y_n)\}$ cходится к $g(y_0) = g(\lim_{n \to \infty} f(x_n))$.
        Следовательно $\lim_{x \to x_0} g(f(x)) = g \left(\lim_{x \to x_0} f(x) \right)$.


    \subsection*{Точки разрыва, их классификация. Разрывы монотонных функций}
        Классификация точек разрыва:
        \begin{itemize}
            \item точка $x_0$ -- устранимая точка разрыва, если $f(x)$ либо не определена, либо не непрерывна в точке $x_0$.
            \item точка $x_0$ -- точка разрыва первого рода, если для $f(x)$ существуют и не равны левый и правый пределы в точке $x_0$.
            Их разность $\omega(x_0) = f(x_0 + 0) - f(x_0 - 0)$ называется скачком функции в точке $x_0$.
            \item точка $x_0$ -- точка разрыва второго рода, если для $f(x)$ не существует конечного левого или правого предела в точке $x_0$.
        \end{itemize}
        \theorem монотонной функция, определенная на \intr{a}{b}, имеет только точки разрыва первого рода.
        \proof пусть $f(x)$ возрастает на \intr{a}{b}. Точек разрыва второго рода нет в силу теоремы о существование односторонних пределов у монотонных функций.
        В силу той же теоремы и неравенстава $f(x - 0) \le f(x_0) \le f(x_0 + 0)$, 
        $f(x - 0) = f(x_0 + 0) = \gamma \neq f(x_0)$ не может выполняться, и следовательно $x_0$ не может быть устранимой точкой разрыва.
        \\
        \\
        \theorem множество точек разрыва определенной на \intr{a}{b} монотонной функции $f(x)$ не более, чем счетно.
        \\
        \proof каждой точке разрыва $x_0$ соответствует интервал \intr{f(x_0 - 0)}{f(x_0 + 0)} точки которого за исключением, быть может, $f(x_0)$,
        не входят в область определения $f(x)$. 
        В силу монотонности $f(x)$ эти интервалы не пересекаются. На каждом интервале выберем рациональную точку.
        Таким образом мы установили взаимнооднозначное соответствие между точками разрыва и подмножеством счетного множества рациональных чисел \Q.
  

    \newpage
    
    \section{}
    \subsection*{Свойства  функций,  непрерывных  на  отрезке}
        \subsubsection*{Ограниченность}
        \theorem если функции $f(x)$ непрерывна на отрезке \otr{a}{b}, то она ограничена на этом отрезке.
        \\
        \proof пусть $f(x)$ неограничена на \otr{a}{b}:
        \[ \forall n \in \mathbb{N} \exists x_n \in [a;b]: |f(x_n)| > n \]
        \[ \{y_n\} = \{f(x_n)\} \]
        последовательность \seq{y_n} -- бесконечно большая. Последовательность \seq{x_n} ограничена, следовательно она содержит в себе сходящуюся подпоследовательность
        \seq{x_{k_n}}. Раз последовательность \seq{x_{k_n}} сходится, а $f(x)$ непрерывна, последовательность \seq{f(x_{k_n})} сходится и не является бесконечно большой,
        что противоречит тому, что \seq{f(x_n)} -- бесконечно большая.
        
        \subsubsection*{Достижимость точных верхней и нижней граней}
        \theorem если функции $f(x)$ непрерывна на отрезке \otr{a}{b}, то она достигает свою точную верхнюю грань $\alpha$ и свою точную нижнюю грань $\beta$.
        \\
        \proof пусть $f(x)$ не достигает $\alpha$. Введем функцию $g(x) = \frac{1}{\alpha - f(x)}$, непрерывную на \otr{a}{b}. Она ограничена сверху $A$.
        \[ \frac{1}{\alpha - f(x)} \le A \]
        \[ f(x) \le \alpha - \frac{1}{A} \]
        следовательно $\alpha - \frac{1}{A}$ -- верхняя грань $f(x)$, что противоречит тому, что $\alpha$ -- точная верхняя грань.
    
    
    \subsection*{Теорема о промежуточных значениях непрерывной функции}
        \theorem если функции $f(x)$ непрерывна на отрезке \otr{a}{b} и принимает на его концах значения разных знаков, то существует такое $x_0 \in (a;b)$, что $f(x_0) = 0$.
        \\
        \proof пусть $f(a) < 0$ и $f(b) > 0$.
        \[ X = \{x \in [a;b]: f(x) < 0\} \neq \varnothing \]
        \[ \exists x_0 = \sup X \]
        так как $f(x)$ непрерывна справа в точке $a$, она сохраняет свой знак в какой-то окрестности $a$, следовательно $x_0 \in (a;b)$.
        Предположим, что $f(x_0) \neq 0$. Тогда в некой $\delta$ окрестности она сохраняет свой знак. Так как $x_0 = \sup X$,
        $\exists x' \in (x_0 - \delta; x_0): f(x') < 0$. Однако $\forall x \in (x_0; x_0 + \delta) \longmapsto f(x) \ge 0$. Следовательно, $f(x_0)$ не сохраняет свой знак
        ни в какой $\delta$ окрестности и равна $0$.
        \\
        \\
        \theorem если функция $f(x)$ непрерывна на отрезке \otr{a}{b}, то она достигает любого своего промежуточного значения.
        \\
        \proof пусть $g(x) = f(x) - \gamma$, где $\gamma$ -- некое промежуточное значение. Функция $g(x)$ непрерывна и на концах \otr{a}{b} принимает значения разных знаков,
        следовательно она достигает $0$ в некой точке $x_0$. $f(x_0) = g(x_0) + \gamma = \gamma$.
        
    \subsection*{Теорема об обратной функции}
        \subsubsection*{Обратная функция}
        Пусть $f(x)$ определена на множестве $X$, $Y$ -- множество ее значений. Если выполняется условие:
        \[ \forall y \in Y \exists! x \in X: f(x) = y \]
        то на $Y$ существует обратная функция $f^{-1}(y)$, каждому $y \in Y$ сопоставляющая такой $x \in X$, что $y = f(x)$.
        
        \subsubsection*{Теорема об обратной функции}
        \lemma если $f(x)$ определена и строго монотонна на $X$, $Y$ -- множество ее значений, тогда на $Y$ определена и имеет такой
        же тип сторогой монотонности, что и $f(x)$, обратная функция $f^{-1}(y)$.
        \\
        \proof так как $f(x)$ сторого монотонна:
        \[ \forall x_1 \ne x_2 \longmapsto f(x_1) < f(x_2) \lor f(x_1) > f(x_2)\]
        Следовательно, на $Y$ существует обратная функция $f^{-1}(y)$. Пусть $f(x)$ возрастает. Докажем возрастание $f^{-1}(y)$:
        \[ \forall y_1 > y_2 \in Y \longmapsto f^{-1}(y_1) = x_1 > x_2 = f^{-1}(y_2) \]
        пусть это не так:
        \[ x_1 \le x_2 \]
        \[ f(x_1) = y_1 \le y_2 = f(x_2) \]
        противоречие.
        \\
        \\
        \theorem пусть на \otr{a}{b} определена, непрерывна и сторого монотонна функция $f(x)$, $\alpha = \inf_{[a;b]} f(x)$, $\beta = \sup_{[a;b]} f(x)$.
        Тогда на \otr{\alpha}{\beta} определена, непрерывна и сторого монотонна в том же направлении, что $f(x)$, обратная функция $f^{-1}(y)$.
        \\
        \proof существование и строгая монотонность $f^{-1}(y)$ следует из леммы. Докажем непрерывность обратной функции для возрастающей $f(x)$.
        Для монотонно возрастающей $f^{-1}(y)$ для всех $y_0 \in (a; b)$ выполняется следующее неравенставо:
        \[ f^{-1}(y_0 - 0) \le f^{-1}(y_0) \le f^{-1}(y_0 + 0) \]
        Пусть разрывна в какой-то точке $y_0 \in (\alpha; \beta)$:
        \[ f^{-1}(y_0 - 0) < f^{-1}(y_0) \lor f^{-1}(y_0) < f^{-1}(y_0 + 0) \]
        Пусть выполняется левое неравенставо $f^{-1}(y_0 - 0) < f^{-1}(y_0)$:
        \[ \forall y \in [\alpha; y_0) \longmapsto a \le f^{-1}(y) \le f^{-1}(y_0 - 0) = \sup_{[\alpha; y_0)} f^{-1}(y) < f^{-1}(y_0) \]
        \[ \forall y \in [y_0; \beta] \longmapsto f^{-1}(y_0) \le f^{-1}(y) = \sup_{[\alpha; y_0)} f^{-1}(y) < f^{-1}(y_0) \]
        значит \intr{f^{-1}(y_0 - 0)}{f^{-1}(y_0)} не принадлежит области значений $f^{-1}(y)$.
        Интервал \intr{f^{-1}(y_0 - 0)}{f^{-1}(y_0)} лежит на отрезке \otr{a}{b}.
        Следовательно образом \otr{\alpha}{\beta} является множество $[a;f^{-1}(y_0 - 0)) \cup (f^{-1}(y_0); b]$, что противоречит тому,
        что образ \otr{\alpha}{\beta} -- \otr{a}{b}.
        \\
        К аналогичному противоречию проходим, рассмотрев правое неравенство $f^{-1}(y_0) < f^{-1}(y_0 + 0)$.
        Для концов отрезка надо аналогичным образом доказать, что $f^{-1}(\alpha + 0) = f^{-1}(\alpha)$ и $f^{-1}(\beta - 0) = f^{-1}(\beta)$.
        
        
    \newpage
    
    \section{}
    \subsection*{Второй замечательный предел}
        \[ e = \lim_{x \to \infty} {\left( 1 + \frac{1}{x} \right)}^x \]


    \subsection*{Определение и свойства экспоненты, показательной функции, логарифмической и степенной функций}
        \subsection*{Степенная функция с рациональным показателем}
        Функцию $f(x) = x^{\frac{p}{q}} = \left( \sqrt[q]{x} \right)^p $ где $x > 0 \in \mathbb{R}, p \in \mathbb{Z}, q \in \mathbb{N}$ назовем степенной с рациональным показателем.
        Если $p = 0$, то $f(x) = 1$, если $p < 0$, то $f(x) = \frac{1}{x^{\frac{-p}{q}}}$.
        \\
        Свойства степенной функции с рациональным показателем $f(x) = x^{\frac{p}{q}} = x^r$:
        \begin{itemize}
            \item $ \left( \sqrt[q]{x} \right)^p = \sqrt[q]{x^p} $
            \item $ r_1, r_2 \in \mathbb{Q} \implies {\left( x^{r_1} \right)}^{r_2} = x^{r_1r_2} $
            \item $ r_1, r_2 \in \mathbb{Q} \implies x^{r_1}x^{r_2} = x^{r_1 + r_2} $
            \item $ x > 1, r > 0 \implies x^r > 1 $
            \item $ x > 1, r_1 < r_2 \in \mathbb{Q} \implies x^{r_1} < x^{r_2}$
        \end{itemize}
        
        \subsection*{Показательная функция}
        Пусть $a > 0, x \in \mathbb{R}$. Последовательность \seq{r_n} сходится к $x$. Определим показательную функцию как 
        \[ f(x) = a^x = \lim_{n \to \infty} r_n \]
        \\
        Свойства показательной функции $f(x) = a^x$:
        \begin{itemize}
            \item $ x_1, x_2 \in \mathbb{R} \implies {\left( a^{x_1} \right)}^{x_2} = a^{x_1x_2} $
            \item $ x_1, x_2 \in \mathbb{R} \implies a^{x_1}a^{x_2} = a^{x_1 + x_2} $
            \item $ a > 1 \implies \lim_{x \to +\infty} a^x = +\infty \land \lim_{x \to -\infty} a^x = 0 $
            \item $ 0 < a < 1 \implies \lim_{x \to +\infty} a^x = 0 \land \lim_{x \to -\infty} a^x = +\infty $
        \end{itemize}
        Если $a = e$, то $f(x)$ называется экспоненциальной функцией, или экспонентой.
        
        \subsection*{Логарифмическая функция}
        По теореме об обратной функции, на интервале \intr{0}{+\infty} определена обратная $ f(x) = a^x$ где $a \in \mathbb{R}, a > 0, a \neq 1 $
        функция $g(x) = \log_a x$.
        \\
        Свойства логарифмической функции:
        \begin{itemize}
            \item $ a > 1 \implies \lim_{x \to +\infty} \log_a x = +\infty \land \lim_{x \to +0} \log_a x = -\infty $
            \item $ 0 < a < 1 \implies \lim_{x \to +\infty} \log_a x = -\infty \land \lim_{x \to +0} \log_a x = +\infty $
        \end{itemize}
        Если $a = e$, то $f(x)$ называется натуральным логарифмом.
        
        \subsection*{Степенная функция с действительным показателем}
        Степенную функцию с действительным показателем определим как $f(x) = x^{\beta} = {\left( a^{\log_a x} \right)}^\beta = a^{\beta \log_a x}$, 
        где $x > 0, a \in \mathbb{R}, a > 0, a \neq 1, \beta \in \mathbb{R}.$


    \subsection*{Непрерывность элементарных функций}
        Элементарные функции непрервны всюду, где определены.
        

    \newpage
    
    \section{}
    \subsection*{Производная функции одной переменной}
        Производной функции $f(x)$ в точке $x_0$ называется следующий предел:
        \[ f'(x_0) = \lim_{x \to x_0} \frac{f(x) - f(x_0)}{x - x_0} \]
        
        
    \subsection*{Односторонние производные}
        Производные справа и слева определяются соответственно как:
        \[ f'(x_0 \pm 0) = \lim_{x \to x_0 \pm 0} \frac{f(x) - f(x_0)}{x - x_0} \]
        
        
    \subsection*{Непрерывность функции, имеющей производную}
        \theorem если функция $f(x)$ дифференцируема в точке $x_0$, то она непрервна в точке $x_0$
        \\
        \proof пусть $f(x)$ дифференцируема в точке $x_0$:
        \[ \Delta f(x_0, \Delta x) = A \Delta x + \alpha(\Delta x) \Delta x = f(x) - f(x_0) \]
        \[ \lim_{\Delta x \to 0} \Delta f(x_0, \Delta x) = \lim_{\Delta x \to 0} \left( f(x_0 + \Delta x) - f(x_0) \right) = 0 \]
        \[ \lim_{\Delta x \to 0} \left( f(x_0 + \Delta x) - f(x_0) \right) = \lim_{\Delta x \to 0} f(x_0 + \Delta x) - \lim_{\Delta x \to 0} f(x_0) =
           \lim_{\Delta x \to 0} f(x_0 + \Delta x) - f(x_0) = 0 \]
        \[ \lim_{\Delta x \to 0} f(x_0 + \Delta x) = f(x_0) \]


    \subsection*{Дифференцируемость функции в точке, дифференциал}
        Функция $f(x)$ дифференцируема в точке $x_0$, если ее приращение $\Delta f(x_0, \Delta x) = f(x + \Delta x) - f(x)$, соответствующее приращению
        аргумента $\Delta x$, можно представить в виде $ \Delta f(x_0, \Delta x) = A \Delta x + \alpha(\Delta x) \Delta x $, где $\alpha(\Delta x)$ -- бесконечно малая при
        $ x \to 0 $.
        \\
        \\
        \theorem понятие дифференцируемости функции $f(x)$ в точке $x_0$ эквивалентно тому, что у функции $f(x)$ существует производная в точке $x_0$.
        \\
        \proof
        Пусть функция $f(x)$ дифференцируема в точке $x_0$:
        \[ \frac{\Delta f(x_0, \Delta x)}{\Delta x} = A + \alpha(\Delta x) \]
        \[ f'(x_0) = \lim_{\Delta x \to 0} \frac{\Delta f(x_0, \Delta x)}{\Delta x} = A \]
        Пусть у функции $f(x)$ есть производная в точке $x_0$:
        \[ f'(x_0) = \lim_{\Delta x \to 0} \frac{\Delta f(x_0, \Delta x)}{\Delta x} \]
        \[ \frac{\Delta f(x_0, \Delta x)}{\Delta x} - f'(x_0) = \alpha(\Delta x) \]
        \[ \lim_{\Delta x \to 0} \alpha(\Delta x) = 0 \]
        \[ \Delta f(x_0, \Delta x) = f'(x_0) \Delta x + \alpha(\Delta x) \Delta x \]


    \subsection*{Геометрический смысл производной и дифференциала}
        Значение производной в точке -- тангенс угла наклона касательной к графику функции в точке.
        
    \subsection*{Производная суммы, произведения и частного двух функций}
        \theorem пусть у функций $f(x)$ и $g(x)$ существуют производные в точке $x_0$. Тогда для функций $f(x) \pm g(x)$, $f(x)g(x)$ и $\frac{f(x)}{g(x)}$ (если $g(x_0) \neq 0$)
        существуют производные.
        \begin{itemize}
            \item $ \left( f(x) \pm g(x) \right)' = f'(x) \pm g'(x) $
            \item $ (f(x)g(x))' = f'(x)g(x) + g'(x)f(x) $
            \item $ \left( \frac{f(x)}{g(x)} \right)' = \frac{f'(x)g(x) - g'(x)f(x)}{g^2(x)} $
        \end{itemize}
        \proof
        \\
        Для суммы и разности:
        \[ (f(x) \pm g(x))' = \lim_{\Delta x \to 0} \frac{\left( f(x + \Delta x) \pm g(x + \Delta x) \right) - \left( f(x) \pm g(x) \right)}{\Delta x} \]
        \[ (f(x) \pm g(x))' = \lim_{\Delta x \to 0} \frac{\left( f(x + \Delta x) - f(x) \right) \pm \left( g(x + \Delta x) -  g(x) \right)}{\Delta x} \]
        \[ (f(x) \pm g(x))' = f'(x) \pm g'(x) \]
        Для произведения функций:
        \[ (f(x)g(x))' = \lim_{\Delta x \to 0} \frac{f(x + \Delta x)g(x + \Delta x) - f(x)g(x)}{\Delta x} \]
        \[ (f(x)g(x))' = \lim_{\Delta x \to 0} \frac{f(x + \Delta x)g(x + \Delta x) - f(x + \Delta x)g(x) + f(x + \Delta x)g(x) - f(x)g(x)}{\Delta x} \]
        \[ (f(x)g(x))' = \lim_{\Delta x \to 0} \frac{f(x + \Delta x)(g(x + \Delta x) - g(x)) + (f(x + \Delta x) - f(x))g(x)}{\Delta x} \]
        \[ (f(x)g(x))' = f(x)g'(x) + f'(x)g(x) \]
        Для отношения функций:
        \[ \left( \frac{f(x)}{g(x)} \right)' = \lim_{\Delta x \to 0} \left( \frac{f(x + \Delta x)}{ g(x + \Delta x) \Delta x } - \frac{f(x)}{ g(x) \Delta x } \right) \]
        \[ \left( \frac{f(x)}{g(x)} \right)' = \lim_{\Delta x \to 0} \left( \frac{f(x + \Delta x) g(x) - f(x) g(x + \Delta x)}{ g(x) g(x + \Delta x) \Delta x } \right) \]
        \[ \left( \frac{f(x)}{g(x)} \right)' = \frac{f'(x) g(x) - f(x) g'(x)}{g^2(x)} \]
        
    \subsection*{Производная сложной функции}
        \theorem Пусть функция $f(x)$ дифференцируема в точке $x_0$, функция $g(y)$ дифференцируема в точке $y_0 = f(x_0)$. Тогда производная функции
        $h(x) = g(f(x))$ в точке $x_0$ равна $h'(x_0) = g'(f(x_0)f'(x_0)$
        \\
        \proof
        \[ \Delta g(y_0, \Delta y) = g'(y_0) \Delta y + \alpha (\Delta y) \Delta y \]
        \[ \lim_{\Delta y \to 0} \alpha (\Delta y) = 0 \]
        \[ \Delta x \to 0 \implies \Delta y \to 0 \]
        \[ \frac{\Delta g(y_0, \Delta y)}{\Delta x} = g'(y_0) \frac{\Delta y}{\Delta x} + \alpha (\Delta y) \frac{\Delta y}{\Delta x} \]
        \[ \lim_{\Delta x \to 0} \frac{\Delta g(y_0, \Delta y)}{\Delta x} = (g(f(x))'(x_0) = g'(y_0)f'(x_0) + 0 = g'(f(x_0))f'(x_0) \]
        
    \subsection*{Производная обратной функции}
        \theorem Пусть функция $f(x)$ непрерывна и строго монотонна на множестве $X = {x: |x - x_0| \le \delta}$ и в точке $x_0$ имеет производную $f'(x_0) \neq 0$.
        Тогда обратная функция $f^{-1}(y)$ в точке $y_0 = f(x_0)$ имеет производную $f'(y_0) = \frac{1}{f(x_0)}$.
        \\
        \proof
        Обратная функция $f^{-1}(y)$ существует по теореме об обратной функции.
        \\
        \[ \Delta y = f(x_0 + \Delta x) - f(x_0) \]
        \[ \Delta x = f^{-1}(y_0 + \Delta y) - f^{-1}(y_0) \]
        \[ \frac{\Delta x}{\Delta y} = \frac{1}{\frac{\Delta y}{\Delta x}} \]
        \[ \Delta x \to 0 \implies \Delta y \to 0 \]
        \[ \lim_{\Delta y \to 0} \frac{\Delta x}{\Delta y} = \lim_{\Delta x \to 0} \frac{1}{\frac{\Delta y}{\Delta x}} \]
        \[ f^{-1}(y_0) = \frac{1}{f'(x_0)} \]
         
        
    \subsection*{Производные элементарных функций}
        \[ (e^x)' = \lim_{\Delta x \to 0} \frac{e^{x + \Delta x} - e^x}{\Delta x} = \lim_{\Delta x \to 0} e^x \frac{e^{\Delta x} - 1}{\Delta x} =
            \lim_{\Delta x \to 0} e^x \frac{1 + \Delta x - 1}{\Delta x} = e^x \]
        \[ (\ln x)' = \lim_{\Delta x \to 0} \frac{ \ln (x + \Delta x) - \ln x} {\Delta x} = \lim_{\Delta x \to 0} \frac {\ln (\frac{x + \Delta x}{x})} {\Delta x} =
           \lim_{\Delta x \to 0} \frac {\Delta x} {x \Delta x} = \frac{1}{x} \]
        \[ (x^{\beta})' = (e^{\beta \ln x})' = \frac{\beta}{x} e^{\beta \ln x} = \beta x^{\beta - 1} \]
        \[ (\sin x)' = \lim_{\Delta x \to 0} \frac{\sin (x + \Delta x) - \sin x} {\Delta x} =
        \lim_{\Delta x \to 0} \frac{2 \sin \left( \frac {\Delta x} {2} \right) \cos \left(x + \frac{\Delta x}{2} \right)} {\Delta x} = \cos x \]
        \[ (\cos x)' = (\sin (x + \frac{\pi}{2}))' = \cos (x + \frac{\pi}{2}) = - \sin x \]
        \[ (\tan x)' = (\frac{\sin x}{\cos x})' = \frac{\cos x \cos x + \sin x \sin x}{\cos^2 x} = \frac{1}{\cos^2 x} = 1 + \tan^2 x \]
        \[ (\arcsin x)' = (\arcsin \left( \sin \varphi \right))' = \frac{1}{\cos \varphi} = \frac{1}{\sqrt{1 - \sin^2 \varphi}} = \frac{1}{\sqrt{1 - x^2}} \]
        \[ (\arccos x)' = (\arcsin \left( \cos \varphi \right))' = \frac{-1}{\sin \varphi} = \frac{-1}{\sqrt{1 - \cos^2 \varphi}} = \frac{-1}{\sqrt{1 - x^2}} \]
        \[ (\arctan x)' = (\arctan \left( \tan \varphi \right))' = \frac{1}{1 + \tan^2 \varphi} = \frac{1}{1 + x^2} \]
        
    \subsection*{Инвариантность формы дифференциала относительно замены переменной}
        \subsubsection*{Определение дифференциала}
        Дифференциалом функции $f(x)$ $df$ в точке $x_0$ назовем линейную относительно $\Delta x$ часть приращения $\Delta f(x_0, \Delta x)$ функции $f(x)$,
        соответствующую приращению аргумента $\Delta x$.
        \[ df (x_0) = f'(x_0) \Delta x \]
        Дифференциалом независимой переменной $x$ $dx$ назовем любое число.
        \[ dx = \Delta x \]
        \[ df (x_0) = f'(x_0) dx \]
        \[ f'(x_0) = \frac{dy(x_0)}{dx} \]
        
        \subsubsection*{Инвариантность формы дифференциала относительно замены переменной}
        Пусть функция $z = g(y)$ дифференцируемая в точке $y_0$:
        \[ dz = g'(y_0) dy \]
        Пусть функция $g(y)$ дифференцируемая в точке $y_0 = f(x_0)$, $f(x)$ дифференцируема в точке $x_0$:
        \[ z = h(x) = g(f(x)) \]
        \[ h'(x_0) = g'(y_0) f'(x_0) \]
        \[ h'(x_0) = \frac{dz}{dx} = g'(y_0) \frac{dy}{dx} \]
        \[ dz = g'(y_0) dy \]
        Вне зависимости от того, является ли $y$ независимой переменной или $y = f(x)$, дифференциал $z = g(y)$ всегда имеет один вид $dz = g'(y_0)dy$.


    \subsection*{Функции, заданные параметрически, их дифференцирование}
        \subsubsection*{Функции, заданные параметрически}
        Пусть функции $x = u(t)$ и $y = v(t)$ определены на множестве $T = [\alpha; \beta]$. Тогда функция $y = f(x)$ задана параметрически, а $t$ -- параметр.

        \subsubsection*{Дифференцирование функций, заданных параметрически}
        \theorem пусть функции $x = u(t)$ и $y = v(t)$ непрерывны и сторого монотонны на множестве $T = [\alpha; \beta]$ и дифференцированы в точке $t_0$, $u'(t_0) \neq 0$.
        Тогда $f'(x_0) = f'(x(t_0)) = \frac{u'(t_0)}{v'(t_0)}$
        \\
        \proof по теореме об обратной функции, существует обратная функция $t = u^{-1}(x)$.
        \[ y = v(t) \]
        \[ t'(x_0) = \frac{1}{u'(t_0)} \]
        \[ y(x_0) = v(t(x_0)) \]
        \[ y'(x_0) = v'(t_0) t'(x_0) = \frac{v'(t_0)}{u'(t_0)} \]


    \newpage
    
    \section{}
    \subsection*{Производные высших порядков}
        Производная порядка $n$ функции $f(x)$ в точке $x_0$ $f^{(n)}(x_0)$ -- производная производной порядка $n - 1$ функции $f(x)$ в точке $x_0$ $f^{(n-1)}(x_0)$.
        \\
        Функция $f(x)$ $n$ раз диффиренцируема в точке $x_0$, если в точке $x_0$ у нее существует производная порядка $n$ $f^{(n)}(x_0)$.


    \subsection*{Формула Лейбница для n-й производной произведения функций}
        \theorem пусть функции $f(x)$ и $g(x)$ $n$ дифференцируемы в точке $x_0$. Тогда производная порядка $n$ их произведения $f(x)g(x)$ в точке $x_0$ $(f(x)g(x))^{(n)}(x_0)$ равна
        $\sum_{k = 0}^{n} {n \choose k} f^{(k)}(x_0) g^{n - k}(x_0)$.
        \\
        \proof докажем по индукции.
        \\
        База: $n = 1$:
        \[ (f(x)g(x))'(x_0) = {n \choose 0} f(x_0)g'(x_0) + {n \choose 1} f'(x_0)g(x_0) = f(x_0)g'(x_0) + f'(x_0)g(x_0) \]
        база верна.
        \\
        Переход: предположим верно для $n$:
        \[ (f(x)g(x))^{(n)}(x_0) = {n \choose 0} f(x_0)g^{(n)}(x_0) + {n \choose 1} f'(x_0)g^{(n-1)}(x_0) + ... + {n \choose n} f^{(n)}(x_0)g(x_0) \]
        Проверим для $n + 1$:
        \[ (f(x)g(x))^{(n + 1)}(x_0) = {n \choose 0} (f(x)g^{(n)}(x))'(x_0) + {n \choose 1} (f'(x)g^{(n-1)}(x))'(x_0) + ... + {n \choose n} (f^{(n)}(x)g(x))'(x_0) \]
        \begin{multline*}
            (f(x)g(x))^{(n + 1)}(x_0) = {n \choose 0} f(x_0)g^{(n+1)}(x_0) + {n \choose 0} f'(x_0)g^{(n)}(x_0) + \\
            {n \choose 1} f'(x_0)g^{(n)}(x_0) + {n \choose 1} f''(x_0)g^{(n-1)}(x_0) + ... + \\
            {n \choose n} (f^{(n)}(x_0)g'(x_0)) + {n \choose n} (f^{(n+1)}(x_0)g(x_0))'
        \end{multline*}
        \[ {n \choose k} + {n \choose k + 1} = {n + 1 \choose k + 1} \]
        \[ {n \choose n} = {n + 1 \choose n + 1} \]
        \[ {n \choose n0} = {n + 1 \choose 0} \]
        \[ (f(x)g(x))^{(n + 1)}(x_0) = {n + 1 \choose 0} f(x_0)g^{(n+1)}(x_0) + {n + 1 \choose 1} f'(x_0)g^{(n)}(x_0) + ... + {n + 1 \choose n + 1} f^{(n+1)}(x_0)g(x_0) \]


    \subsection*{Дифференциал второго порядка}
        Пусть функция $f(x)$ дифференципуема в $\Delta$ окрестности точки $x_0$ и дважды дифференцируема в точке $x_0$.
        Тогда в $\Delta$ окрестности точки $x_0$ определена функция $dy = f'(x)dx$.
        Дифференциал второго порядка функции $f(x)$ в точке $x_0$ -- дифференциал ее дифференциала $dy$ в точке $x_0$ $d^2f(x_0)$.


    \subsection*{Отсутствие инвариантности формы дифференциал второго порядка относительно замены переменной}
        Пусть $z = g(y)$. Тогда:
        \[ dz = g'(y_0) dy \]
        \[ d(dz) (y_0) = d(g'(y)dy) (y_0) = (d(g'(y))dy + g'(y)d(dy))) (y_0) = g''(y_0)(dy)^2 + 0 = g''(y_0)(dy)^2 \]
        Пусть $z = g(y)$, $y = f(x)$. Тогда:
        \[ dz (y_0) = g'(y_0) dy \]
        \[ d(dz) (y_0) = d(g'(y) dy) (y_0) = (d(g'(y))dy + g'(y)d(dy)) (y_0) = g''(y_0)(dy)^2 + g'(y_0)d^2y \]
        Как видно, дифференциал второго порядка не обладает инвариантностью формы.


    \newpage

    \section{}
    \subsection*{Теорема Ферма (необходимое условие существования локального экстремума)}
        Пусть $f(x)$ определена в точке $x_0$ и в некой $\Delta$ окрестности точки $x_0$.

        \subsubsection*{Локальный экстремум}
        Точка $x_0$ называется локальным минимум функции $f(x)$, если есть такая выколотая $\delta \le \Delta$ окрестность точки $x_0$, что для любого $x$
        в этой окрестности $f(x) > f(x_0)$.
        Точка $x_0$ называется локальным максимумом функции $f(x)$, если есть такая выколотая $\delta \le \Delta$ окрестность точки $x_0$, что для любого $x$
        в этой окрестности $f(x) < f(x_0)$.
        Точка $x_0$ называется локальным экстремумом функции $f(x)$, она является либо локальным минимумом, либо локальным максимумом.
        
        \subsubsection*{Возрастание и убывание функции в точке}
        Если функций $f(x)$ возрастает в точке $x_0$, то существует такая $\delta \le \Delta$ окрестности точки $x_0$,
        что для всех $x < x_0$ $f(x) < f(x_0)$ и для всех $x > x_0$ $f(x) > f(x_0)$.
        \\
        Если функций $f(x)$ убывает в точке $x_0$, то существует такая $\delta \le \Delta$ окрестности точки $x_0$,
        что для всех $x < x_0$ $f(x) > f(x_0)$ и для всех $x > x_0$ $f(x) < f(x_0)$.
        
        \subsubsection*{Возрастание и убывание дифференциремой функции в точке}
        \theorem пусть функция $f(x)$ дифференцируема в точке $x_0$. Если $f'(x_0) < 0$, то $f(x)$ убывает в точке $x_0$.
        Если $f'(x_0) > 0$, то $f(x)$ возрастает в точке $x_0$.
        \\
        \proof пусть $f'(x_0) > 0$.
        \[ f'(x_0) = \lim_{\Delta x \to 0} \frac{f(x_0 + \Delta x) - f(x_0)}{\Delta x} \]
        \[ \forall \varepsilon > 0 \exists \delta = \delta(\varepsilon) > 0: \forall x: 0 < |x - x_0| < \delta \longmapsto
        f'(x_0) - \varepsilon < \frac{f(x) - f(x_0)}{x - x_0} < f'(x_0) + \varepsilon \]
        Пусть $\varepsilon < f'(x_0)$. Тогда из предыдущего неравенства следует возрастание $f(x)$.
        \\
        Аналогично доказывается для  $f'(x_0) < 0$.

        \subsubsection*{Теорема Ферма}
        \theorem если функция $f(x)$ дифференицируема в точке $x_0$ и точка $x_0$ является локальным экстремумом функции $f(x)$, то $f'(x_0) = 0$.
        \\
        \proof Если $f'(x_0) \ne 0$, то $f(x)$ возрастает или убывает в точке $x_0$, что противоречит определению локального экстремума.
        Следовательно, $f'(x_0) = 0$.


    \subsection*{Теоремы о среднем Ролля, Лагранжа, Коши}
        \subsubsection*{Теорема Ролля}
        \theorem пусть функция $f(x)$ непрерывна на \otr{a}{b}, дифференцируема на \intr{a}{b}, и $f(a) = f(b)$.
        Тогда найдется такая точка $x_0$ на \intr{a}{b}, что $f'(x_0) = 0$.
        \\
        \proof так как $f(x)$ непрерывна на \otr{a}{b}, она достигает своей точной верхней грани $\alpha$ и точной нижней грани $\beta$.
        Если $\alpha = \beta$, то $f(x) = const$ и $f'(x) = 0 \forall x \in (a;b)$.
        Иначе, так как $f(a) = f(b)$, найдется такая точка $x_0$, что $f(x_0) = \alpha$ или $f(x_0) = \beta$.
        По теорема Ферма в $x_0$ $f'(x_0) = 0$.
        
        \subsubsection*{Теорема Лагранжа}
        \theorem пусть функция $f(x)$ непрерывна на \otr{a}{b} и дифференцируема на \intr{a}{b}.
        Тогда найдется такая точка $x_0$ на \intr{a}{b}, что $ f(b) - f(a) = f'(x_0) (b - a) $.
        \\
        \proof введем следующую функцию:
        \[ F(x) = f(x) - f(a) - \frac{f(b) - f(a)}{b - a}(x - a) \]
        \[ F'(x) = f'(x) - \frac{f(b) - f(a)}{b - a} \]
        $F(x)$ непрерывна на \otr{a}{b} и дифференцируема на \intr{a}{b} , $F(a) = F(b) = 0$. Следовательно, по теореме Ролля существует такая точка
        $x_0$, что $F'(x_0) = 0$. Тогда:
        \[ F'(x) = 0 = f'(x) - \frac{f(b) - f(a)}{b - a} \]
        \[ f(b) - f(a) = f'(x_0) (b - a) \]

        \subsubsection*{Теорема Коши}
        \theorem пусть функции $f(x)$ и $g(x)$ непрерывны на \otr{a}{b} и дифференцируемы на \intr{a}{b}, $g'(x)$ не обращается на \intr{a}{b} в $0$.
        Тогда найдется такая точка $x_0$ на \intr{a}{b}, что $ \frac{f(b) - f(a)}{g(b) - g(a)} = \frac{f'(x_0)}{g'(x_0} $.
        \\
        \proof введем следующую функцию:
        \[ F(x) = f(x) - f(a) - \frac{f(b) - f(a)}{g(b) - g(a)}(g(x) - g(a)) \]
        \[ F'(x) = f'(x) - \frac{f(b) - f(a)}{g(b) - g(a)} g'(x) \]
        $F(x)$ непрерывна на \otr{a}{b} и дифференцируема на \intr{a}{b} , $F(a) = F(b) = 0$. Следовательно, по теореме Ролля существует такая точка
        $x_0$, что $F'(x_0) = 0$. Тогда:
        \[ F'(x) = 0 = f'(x) - \frac{f(b) - f(a)}{g(b) - g(a)} g'(x) \]
        \[ \frac{f(b) - f(a)}{g(b) - g(a)} = \frac{f'(x_0)}{g'(x_0} \]
        

    \subsection*{Формула Тейлора с остаточным членом в формах Пеано и Лагранжа}
    \subsection*{Основные разложения по формуле Тейлора}
    \subsection*{Правило Лопиталя для раскрытия неопределенностей вида $\frac{0}{0}$}
    \subsection*{Правило Лопиталя для раскрытия неопределенностей вида $\frac{\infty}{\infty}$} 

\end{document}
